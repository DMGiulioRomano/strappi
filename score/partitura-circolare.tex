\documentclass[tikz, a3paper, landscape, border=0]{standalone}
\usetikzlibrary{calc}
\usetikzlibrary{intersections}

\usepackage{comandi}
\usepackage{musicography}
\usepackage{pgfmath}
\usepackage{hyperref}  
\usetikzlibrary{decorations.markings}
\usetikzlibrary {decorations.text}
\catcode`\|12
\usetikzlibrary { decorations.pathmorphing, decorations.pathreplacing, decorations.shapes}
 
\begin{document}
\begin{tikzpicture}







%======================================================================
% ------------------------------ preparazione foglio di lavoro ----------------------------------------------------
\def\paperHeight{29.7}
\def\paperWidth{42}
\fill[white] (0,0) rectangle ++ (\paperWidth,\paperHeight);
% Imposta la griglia del foglio A3 come riferimento (42x29.7 cm)
%\draw[step=.1cm, gray, very thin] (0, 0) grid (\paperWidth, \paperHeight); % Griglia per visualizzare lo spazio A3
%\draw[step=1cm, black, very thin] (0, 0) grid (\paperWidth, \paperHeight); % Griglia per visualizzare lo spazio A3
%======================================================================









%======================================================================
% Posizionamento delle carte in cerchio
\def\circleRadius{11cm}  % Regola questo valore in base alle tue esigenze
\def\numCards{11}  % Numero di carte nel cerchio principale
\def\angleOffset{30}  % Offset angolare per ruotare l'intero layout
% ------------------
\def\pointWidth{3pt}
% ------------------
\def\minimumSize{1.5pt}
% ------------------
\def\arrowWidth{6pt}
\def\arrowLength{9pt}
\def\sumWidth{1pt}
\def\lineWidthMore{3pt}
\def\lineWidthMedium{\lineWidthMore-1.5}
\def\lineWidth{\lineWidthMore-2.5pt}
\def\arrowWidthMore{9pt}
\def\arrowLengthMore{12pt}
\def\sumWidthMore{3pt}
\def\roundedCorners{8pt}
% ------------------
\def\cardRadius{2.1cm} 
\def\cardRadiusTwo{\cardRadius + .5cm} 
% ------------------
\def\battenteheight{.25}
\def\battentewidth{1.3}
%======================================================================





%======================================================================
% -----------------------------	definizione cards -------------------------------------------------------------------
%
\foreach \i in {1,...,3} {
    \coordinate (card\i) at ($(4,\paperHeight) - (0,6*\i-1)$);
    \gettikzxy{(card\i)}{\bx}{\by}
    \cardtwo{\bx}{\by}{\cardRadius}{card\i}{\cardRadiusTwo}
    }
\foreach \i in {4,...,\numCards} {
    \placeCardOnCircle{\i}{\circleRadius}{\numCards}{\angleOffset}
    
}
    \gettikzxy{(card1)}{\bx}{\by}
    \draw [line width = \lineWidthMore-1.5pt,name path=card1] (\bx, \by) circle (\cardRadiusTwo-4);
    \gettikzxy{(card2)}{\bx}{\by}
    \draw [line width = \lineWidthMore-1.5pt,name path=card2] (\bx, \by) circle (\cardRadiusTwo-4);
    \gettikzxy{(card3)}{\bx}{\by}
    \draw [line width = \lineWidthMore-1.5pt, name path=card3] (\bx, \by) circle (\cardRadiusTwo-4);
    \gettikzxy{(card4)}{\bx}{\by}
    
\circleAroundReversed{\bx}{\by}{\cardRadius}{card4}{\cardRadiusTwo}{1}{-25}{\lineWidthMore}
    \gettikzxy{(card5)}{\bx}{\by}
\circleAroundReversed{\bx}{\by}{\cardRadius}{card5}{\cardRadiusTwo}{.3}{20}{\lineWidthMedium}
    \gettikzxy{(card6)}{\bx}{\by}
\circleAroundReversed{\bx}{\by}{\cardRadius}{card6}{\cardRadiusTwo}{1}{-25}{\lineWidthMore}
    \gettikzxy{(card7)}{\bx}{\by}
\circleAroundReversed{\bx}{\by}{\cardRadius}{card7}{\cardRadiusTwo}{.3}{110}{\lineWidthMedium}
    \gettikzxy{(card8)}{\bx}{\by}
\circleAroundReversed{\bx}{\by}{\cardRadius}{card8}{\cardRadiusTwo}{.3}{160}{\lineWidthMedium}
    \gettikzxy{(card9)}{\bx}{\by}
\circleAroundReversed{\bx}{\by}{\cardRadius}{card9}{\cardRadiusTwo}{1}{-30}{\lineWidthMore}
    \gettikzxy{(card10)}{\bx}{\by}
\circleAroundReversed{\bx}{\by}{\cardRadius}{card10}{\cardRadiusTwo}{.3}{-100}{\lineWidthMedium}
    \gettikzxy{(card11)}{\bx}{\by}
\circleAroundReversed{\bx}{\by}{\cardRadius}{card11}{\cardRadiusTwo}{.3}{-70}{\lineWidthMedium}
 
    \coordinate (card12) at (\paperWidth/2,\paperHeight/2);
    \gettikzxy{(card12)}{\bx}{\by}
    \cardtwo{\bx}{\by}{\cardRadius}{card12}{\cardRadiusTwo}
\circleAround{\bx}{\by}{\cardRadius}{card12}{\cardRadiusTwo}
%======================================================================







%======================================================================
% -----------------------------	definizione campane		-------------------------------------------------------
%
\campana{0}{-.3}{.8}{capovolta}{libera}{card1}
\campana{0}{-.3}{.8}{capovolta}{libera}{card2}
\campana{0}{-.3}{.8}{ordinaria}{bloccata}{card3}
\campana{0}{-.3}{.8}{ordinaria}{bloccata}{card4}
\campana{0}{-.3}{.8}{ordinaria}{libera}{card5}
\campana{0}{-.3}{.8}{ordinaria}{libera}{card6}
%\campanaAcc{0}{0}{.8}{bloccata}{card6bis}
\campanaAccZ{0}{-.3}{.8}{libera}{card7}
\campana{0}{-.3}{.8}{ordinaria}{libera}{card8}
\campana{0}{-.3}{.8}{ordinaria}{libera}{card9}
\campanaAccZ{0}{-.3}{.8}{libera}{card10}
\campanaAccZ{0}{-.3}{.8}{libera}{card11}
\campana{0}{-.3}{.8}{ordinaria}{bloccata}{card12}
%======================================================================








%======================================================================
% -----------------------------	definizione battenti		-------------------------------------------------------
%
% ----------------------------	CARD	1
\updateBattentePosition{0.1}{0}
\battente{card1}{\battenteStartX}{\battenteStartY}{\battentewidth}{\battenteheight}{libero}{60}
\strofinato{card1}{.65}{.1}{-1.185}{.4}
% ----------------------------	CARD	2
\updateBattentePosition{0}{.23}
\battente{card2}{\battenteStartX}{\battenteStartY}{\battentewidth}{\battenteheight}{libero}{0}
\draw[arrows = {-Latex}, double distance=1.2pt] ($(card2) + (0,1.2)$) -- ($(card2) + (0,.33)$);
% ----------------------------	CARD	3
\updateBattentePosition{0}{.02}
\battente{card3}{\battenteStartX}{\battenteStartY}{\battentewidth}{\battenteheight}{libero}{60}
\strofinato{card3}{.8}{.1}{-1.45}{.4}
% ----------------------------	CARD	4
\updateBattentePosition{-.35}{.02}
\battente{card4}{\battenteStartX}{\battenteStartY}{\battentewidth}{\battenteheight}{libero}{45}
% ----------------------------	CARD	5
\updateBattenteAccPosition{0}{.23}
\battenteAcc{card5}{\battenteStartAccX}{\battenteStartAccY}{\battentewidth}{\battenteheight}{aperto}{0}
\draw[arrows = {-Latex}, double distance=1.2pt] ($(card5) + (0,1.2)$) -- ($(card5) + (0,.33)$);
% ----------------------------	CARD	6
\updateBattentePosition{-.45}{-.3}
\battente{card6}{\battenteStartX}{\battenteStartY}{\battentewidth}{\battenteheight}{bloccato}{0}
% ----------------------------	CARD	7
\updateBattentePosition{0}{.02}
\battente{card7}{\battenteStartX}{\battenteStartY}{\battentewidth}{\battenteheight}{bloccato}{60}
\strofinato{card7}{.8}{.1}{-1.45}{.4}
% ----------------------------	CARD	8
\updateBattentePosition{-.35}{.02}
\battente{card8}{\battenteStartX}{\battenteStartY}{\battentewidth}{\battenteheight}{bloccato}{45}
% ----------------------------	CARD	9
\updateBattentePosition{0}{.02}
\battente{card9}{\battenteStartX}{\battenteStartY}{\battentewidth}{\battenteheight}{libero}{60}
\strofinato{card9}{.8}{.1}{-1.45}{.4}
% ----------------------------	CARD	10
\updateBattentePosition{-.45}{-.3}
\battente{card10}{\battenteStartX}{\battenteStartY}{\battentewidth}{\battenteheight}{libero}{0}
% ----------------------------	CARD	11
\updateBattentePosition{-.35}{.02}
\battente{card11}{\battenteStartX}{\battenteStartY}{\battentewidth}{\battenteheight}{bloccato}{45}
% ----------------------------	CARD	12
\updateBattentePosition{0.36}{-.4}
\battente{card12}{\battenteStartX}{\battenteStartY}{\battentewidth}{\battenteheight}{libero}{80}
\draw[line width = .8pt] ($(card12)+(-.7,+.085)$) -- ($(card12)+(+.7,+.085)$);
\draw[->]   ($(card12)+(-.5,-.7)$) to[out=-30, in=200, looseness=1] ($(card12)+(.6,-.7)$);
\campanaSbiadita{0}{-.3}{.8}{ordinaria}{bloccata}{card12}
%======================================================================









%======================================================================
% -----------------------------	definizione durate e dinamiche -------------------------------------------------
% 
%
% ----------------------------	CARD	1
\node (dur) at ($(card1) + ({\cardRadius*cos(90)},{\cardRadius*sin(50)})$) {\normalsize 30 s $\sim$};
\node (amp) at ($(card1) + ({\cardRadius*cos(90)},{-\cardRadius*sin(40)})$) {\normalsize \textit{mf} $\sim$ \textit{ff}};
% ----------------------------	CARD	2
\node (dur) at ($(card2) + ({\cardRadius*cos(90)},{\cardRadius*sin(50)})$) {\normalsize epsylon};
\node (amp) at ($(card2) + ({\cardRadius*cos(90)},{-\cardRadius*sin(40)})$) {\normalsize \textit{mf}};
% ----------------------------	CARD	3
\node (dur) at ($(card3) + ({\cardRadius*cos(90)},{\cardRadius*sin(50)})$) {\normalsize 30 s $\sim$};
\node (amp) at ($(card3) + ({\cardRadius*cos(90)},{-\cardRadius*sin(40)})$) {\normalsize \textit{mf} $\sim$ \textit{f}};
% ----------------------------	CARD	4
\node (dur) at ($(card4) + ({\cardRadius*cos(90)},{\cardRadius*sin(50)})$) {\normalsize eps $\sim$ 2 s};
\node (amp) at ($(card4) + ({\cardRadius*cos(90)},{-\cardRadius*sin(40)})$) {\normalsize \textit{mp} $\sim$ \textit{mf}};
% ----------------------------	CARD	5
\node (dur) at ($(card5) + ({\cardRadius*cos(90)},{\cardRadius*sin(50)})$) {\normalsize epsylon};
\node (amp) at ($(card5) + ({\cardRadius*cos(90)},{-\cardRadius*sin(40)})$) {\normalsize \textit{pp} $\sim$ \textit{mf}};
% ----------------------------	CARD	6
\node (dur) at ($(card6) + ({\cardRadius*cos(90)},{\cardRadius*sin(50)})$) {\normalsize 1 $\sim$ 10 s};
\node (amp) at ($(card6) + ({\cardRadius*cos(90)},{-\cardRadius*sin(40)})$) {};
\gettikzxy{(amp)}{\ampx}{\ampy}
\coordinate (nulla) at ($(\ampx-\cardRadius/4, \ampy)$);
\coordinate (B) at ($(nulla) + (\cardRadius/2.5, \cardRadius/35)$);
\coordinate (C) at ($(nulla) + (\cardRadius/2.5, -\cardRadius/35)$);
\coordinate (f) at ($(\ampx+\cardRadius/4, \ampy)$);
\node[draw=black, circle,  thin, inner sep=1pt] at (nulla) {};
    \draw[thin] ($(nulla)+(.05,0)$) -- (B);  % Linea superiore del crescendo
    \draw[thin] ($(nulla)+(.05,0)$) -- (C); % Linea inferiore del crescendo
	\node at (f) {\normalsize \textit{f}};

% ----------------------------	CARD	7
\node (dur) at ($(card7) + ({\cardRadius*cos(90)},{\cardRadius*sin(50)})$) {\normalsize 3 $\sim$ 9 s};
\node (amp) at ($(card7) + ({\cardRadius*cos(90)},{-\cardRadius*sin(40)})$) {};
\gettikzxy{(amp)}{\ampx}{\ampy}
\coordinate (mf) at ($(\ampx-\cardRadius/4, \ampy)$);
\coordinate (B) at ($(mf) + (\cardRadius/2.5, \cardRadius/35)$);
\coordinate (C) at ($(mf) + (\cardRadius/2.5, -\cardRadius/35)$);
\coordinate (f) at ($(\ampx+\cardRadius/4, \ampy)$);
\node at ($(mf)$) {\normalsize\textit{mf}};
    \draw[thin] ($(mf)+(.3,0)$) -- (B);  % Linea superiore del crescendo
    \draw[thin] ($(mf)+(.3,0)$) -- (C); % Linea inferiore del crescendo
	\node at (f) {\normalsize \textit{ff}};
% ----------------------------	CARD	8
\node (dur) at ($(card8) + ({\cardRadius*cos(90)},{\cardRadius*sin(50)})$) {\normalsize eps $\sim$ 4 s};
\node (amp) at ($(card8) + ({\cardRadius*cos(90)},{-\cardRadius*sin(40)})$) {\normalsize \textit{mf}};
% ----------------------------	CARD	9
\node (dur) at ($(card9) + ({\cardRadius*cos(90)},{\cardRadius*sin(50)})$) {\normalsize 4 $\sim$ 12 s};
\node (amp) at ($(card9) + ({\cardRadius*cos(90)},{-\cardRadius*sin(40)})$) {};
\gettikzxy{(amp)}{\ampx}{\ampy}
\coordinate (mf) at ($(\ampx-\cardRadius/4, \ampy)$);
\coordinate (B) at ($(mf) + (\cardRadius/2.5, \cardRadius/35)$);
\coordinate (C) at ($(mf) + (\cardRadius/2.5, -\cardRadius/35)$);
\coordinate (sfz) at ($(\ampx+\cardRadius/3.5, \ampy)$);
\node at ($(mf)$) {\normalsize \textit{mf}};
    \draw[thin] ($(mf)+(.3,0)$) -- (B);  % Linea superiore del crescendo
    \draw[thin] ($(mf)+(.3,0)$) -- (C); % Linea inferiore del crescendo
	\node at (sfz) {\normalsize \textit{sfz}};
% ----------------------------	CARD	10
\node (dur) at ($(card10) + ({\cardRadius*cos(90)},{\cardRadius*sin(50)})$) {\normalsize eps $\sim$ 4 s};
\node (amp) at ($(card10) + ({\cardRadius*cos(90)},{-\cardRadius*sin(40)})$) {\normalsize \textit{mp} $\sim$ \textit{f}};
% ----------------------------	CARD	11
\node (dur) at ($(card11) + ({\cardRadius*cos(90)},{\cardRadius*sin(50)})$) {\normalsize eps $\sim$ 3 s};
\node (amp) at ($(card11) + ({\cardRadius*cos(90)},{-\cardRadius*sin(40)})$) {\normalsize \textit{p} $\sim$ \textit{ff}};
% ----------------------------	CARD	12
\node (dur) at ($(card12) + ({\cardRadius*cos(90)},{\cardRadius*sin(50)})$) {\normalsize 10 $\sim$ 30 s};
\node (amp) at ($(card12) + ({\cardRadius*cos(90)},{-\cardRadius*sin(40)})$) {\normalsize \textit{pp} $\sim$ \textit{p}};
%======================================================================












%======================================================================
% -----------------------------	definizione FRECCE e NODI carte --------------------------------------------
%	
% ----------------------------	CARD	1 - 2
\node at ($(card1)!\cardRadiusTwo-4!(card2)$) [] (out1) {};
\node at ($(card2)!\cardRadiusTwo-4!(card1)$) [] (in2) {};
\draw[line width = \lineWidthMore] decorate[decoration={markings, mark=between positions 0.6 and 1 step 10mm with {\arrow[scale=1.3]{stealth}}}] {(out1) -- (in2.center)};
\draw[line width = \lineWidthMedium] (out1.center) -- (in2.center);
% ----------------------------	CARD	2 - 3
\node at ($(card2)!\cardRadiusTwo-4!(card3)$) [] (out2) {};
\node at ($(card3)!\cardRadiusTwo-4!(card2)$) [] (in3) {};
\draw[line width = \lineWidthMore] decorate[decoration={markings, mark=between positions 0.6 and 1 step 10mm with {\arrow[scale=1.3]{stealth}}}] {(out2) -- (in3.center)};
\draw[line width = \lineWidthMedium] (out2.center) -- (in3.center);
% ----------------------------	CARD	3 - 4
\node at ($(card3)!\cardRadiusTwo-4!(card4)$) [] (out3) {};
\node at ($(card4)!\cardRadiusTwo!(card3)$) [] (in4) {};
\draw[line width = \lineWidthMore] decorate[decoration={markings, mark=between positions 0.3 and 1 step 12mm with {\arrow[scale=1.3]{stealth}}}] {(out3.center) -- (in4.center)};
\draw[line width = \lineWidthMedium] (out3.center) -- (in4.center);
% ----------------------------	CARD	4 - 5
\node at ($(card4)!\cardRadiusTwo!(card5)$) [] (out4) {};
\node at ($(card5)!\cardRadiusTwo!(card4)$) [] (in5) {};
\draw[line width = \lineWidthMore] decorate[decoration={markings, mark=between positions 0.3 and .9 step 11mm with {\arrow[scale=1.3]{stealth}}}] {(out4.center) -- (in5.center)};
\draw[line width = \lineWidthMedium] (out4.center) -- (in5.center);
% ----------------------------	CARD	5 - 12
\node at ($(card5)!\cardRadiusTwo!(card12)$) [] (out5) {};
\node at ($(card12)!\cardRadiusTwo!(card5)$) [] (in12-5) {};
\draw[line width = \lineWidthMore] decorate[decoration={markings, mark=between positions 0.3 and .9 step 14mm with {\arrow[scale=1.3]{stealth}}}] {(out5.center) -- (in12-5.center)};
\draw[line width = \lineWidthMedium] (out5.center) -- (in12-5.center);
% ----------------------------	CARD	5 - 6 *********
\node at ($(card5)!\cardRadiusTwo!(card6)$) [] (out5) {};
\node at ($(card6)!\cardRadiusTwo!(card5)$) [] (in6) {};
\draw[line width = \lineWidth] decorate[decoration={markings, mark=between positions 0.3 and 0.9 step 11mm with {\arrow[scale=1.5]{stealth}}}] {(out5.center) -- (in6.center)};
\draw[line width = \lineWidth]  (out5.center) -- (in6.center);
% ----------------------------	CARD	12 - 6
\node at ($(card12)!\cardRadiusTwo!(card6)$) [] (out12-6) {};
\node at ($(card6)!\cardRadiusTwo!(card12)$) [] (in6) {};
\draw[line width = \lineWidthMore] decorate[decoration={markings, mark=between positions 0.3 and .9 step 14mm with {\arrow[scale=1.3]{stealth}}}] { (out12-6.center) -- (in6.center)};
\draw[line width = \lineWidthMedium]  (out12-6.center) -- (in6.center);
% ----------------------------	CARD	6 - 7
\node at ($(card6)!\cardRadiusTwo!(card7)$) [] (out6) {};
\node at ($(card7)!\cardRadiusTwo!(card6)$) [] (in7) {};
\draw[line width = \lineWidthMore] decorate[decoration={markings, mark=between positions 0.3 and .9 step 11mm with {\arrow[scale=1.3]{stealth}}}] { (out6.center) -- (in7.center)};
\draw[line width = \lineWidthMedium]  (out6.center) -- (in7.center);
% ----------------------------	CARD	7 - 8
\node at ($(card7)!\cardRadiusTwo!(card8)$) [] (out7) {};
\node at ($(card8)!\cardRadiusTwo!(card7)$) [] (in8) {};
\draw[line width = \lineWidthMore] decorate[decoration={markings, mark=between positions 0.3 and .9 step 11mm with {\arrow[scale=1.3]{stealth}}}] { (out7.center) -- (in8.center)};
\draw[line width = \lineWidthMedium]  (out7.center) -- (in8.center);
% ----------------------------	CARD	8 - 12
\node at ($(card8)!\cardRadiusTwo!(card12)$) [] (out8) {};
\node at ($(card12)!\cardRadiusTwo!(card8)$) [] (in12-8) {};
\draw[line width = \lineWidthMore] decorate[decoration={markings, mark=between positions 0.3 and .9 step 14mm with {\arrow[scale=1.3]{stealth}}}] { (out8.center) -- (in12-8.center)};
\draw[line width = \lineWidthMedium]  (out8.center) -- (in12-8.center);
% ----------------------------	CARD	8 - 9 *********
\node at ($(card8)!\cardRadiusTwo!(card9)$) [] (out8) {};
\node at ($(card9)!\cardRadiusTwo!(card8)$) [] (in9) {};
\draw[line width = \lineWidth] decorate[decoration={markings, mark=between positions 0.3 and 0.9 step 11mm with {\arrow[scale=1.5]{stealth}}}] {  (out8.center) -- (in9.center)};
\draw[line width = \lineWidth]   (out8.center) -- (in9.center);
% ----------------------------	CARD	12 - 9
\node at ($(card12)!\cardRadiusTwo!(card9)$) [] (out12-9) {};
\node at ($(card9)!\cardRadiusTwo!(card12)$) [] (in9) {};
\draw[line width = \lineWidthMore] decorate[decoration={markings, mark=between positions 0.3 and .9 step 14mm with {\arrow[scale=1.3]{stealth}}}] { (out12-9.center) -- (in9.center)};
\draw[line width = \lineWidthMedium]  (out12-9.center) -- (in9.center);
% ----------------------------	CARD	9 - 10
\node at ($(card9)!\cardRadiusTwo!(card10)$) [] (out9) {};
\node at ($(card10)!\cardRadiusTwo!(card9)$) [] (in10) {};
\draw[line width = \lineWidthMore] decorate[decoration={markings, mark=between positions 0.3 and .9 step 11mm with {\arrow[scale=1.3]{stealth}}}] { (out9.center) -- (in10.center)};
\draw[line width = \lineWidthMedium]  (out9.center) -- (in10.center);
% ----------------------------	CARD	10 - 11
\node at ($(card10)!\cardRadiusTwo!(card11)$) [] (out10) {};
\node at ($(card11)!\cardRadiusTwo!(card10)$) [] (in11) {};
\draw[line width = \lineWidthMore] decorate[decoration={markings, mark=between positions 0.3 and .9 step 11mm with {\arrow[scale=1.3]{stealth}}}] { (out10.center) -- (in11.center)};
\draw[line width = \lineWidthMedium]  (out10.center) -- (in11.center);
% ----------------------------	CARD	11 - 12
\node at ($(card11)!\cardRadiusTwo!(card12)$) [] (out11) {};
\node at ($(card12)!\cardRadiusTwo!(card11)$) [] (in12-11) {};
\draw[line width = \lineWidthMore] decorate[decoration={markings, mark=between positions 0.3 and .9 step 14mm with {\arrow[scale=1.3]{stealth}}}] { (out11.center) -- (in12-11.center)};
\draw[line width = \lineWidthMedium]  (out11.center) -- (in12-11.center);

% ----------------------------	CARD	11 - 4 *********
\node at ($(card11)!\cardRadiusTwo!(card4)$) [] (out11) {};
\node at ($(card4)!\cardRadiusTwo!(card11)$) [] (in4) {};
\draw[line width = \lineWidth] decorate[decoration={markings, mark=between positions 0.3 and 0.9 step 11mm with {\arrow[scale=1.5]{stealth}}}] { (out11.center) -- (in4.center)};
\draw[line width = \lineWidth]  (out11.center) -- (in4.center);
% ----------------------------	CARD	12 - 4
\node at ($(card12)!\cardRadiusTwo!(card4)$) [] (out12-4) {};
\node at ($(card4)!\cardRadiusTwo!(card12)$) [] (in4-12) {};
\draw[line width = \lineWidthMore] decorate[decoration={markings, mark=between positions 0.3 and .9 step 14mm with {\arrow[scale=1.3]{stealth}}}] { (out12-4.center) -- (in4-12.center)};
\draw[line width = \lineWidthMedium]  (out12-4.center) -- (in4-12.center);
%======================================================================















%======================================================================
% -----------------------------	definizione FRECCE superiori --------------------------------------------------
%
% ----------------------------	CARD 5 => CARD 4
	\draw[opacity=0,name path=segment54] (card5) to[out=100, in=0, looseness=2] (card4);
	\path[name intersections={of=card5 and segment54, by={card55}}];
	\node[opacity=0,fill=black, circle, minimum size=5pt, inner sep=0pt] at (card55) {};
	\path[name intersections={of=card4 and segment54, by={card44}}];
	\node[opacity=0,fill=black, circle, minimum size=5pt, inner sep=0pt] at (card44) {};
	\draw[line width = \lineWidthMore] decorate[decoration={markings, mark=between positions 0.2 and .9 step 11mm with {\arrow[scale=1.3]{stealth}}}] {(card55.center) to[out=100, in=0, looseness=2] (card44.center)};
	\draw[line width = \lineWidthMore] (card55.center) to[out=100, in=0, looseness=2] (card44.center);
% ----------------------------	CARD 8 => CARD 7
	\draw[opacity=0,name path=segment87] (card8) to[out=235, in=135, looseness=2] (card7);
	\path[name intersections={of=card8 and segment87, by={card88}}];
	\node[opacity=0,fill=black, circle, minimum size=5pt, inner sep=0pt] at (card88) {};
	\path[name intersections={of=card7 and segment87, by={card77}}];
	\node[opacity=0,fill=black, circle, minimum size=5pt, inner sep=0pt] at (card77) {};
	\draw[line width = \lineWidthMore] decorate[decoration={markings, mark=between positions 0.2 and .9 step 11mm with {\arrow[scale=1.3]{stealth}}}] {(card88.center) to[out=235, in=135, looseness=2] (card77.center)};
	\draw[line width = \lineWidthMore] (card88.center) to[out=235, in=135, looseness=2] (card77.center);
% ----------------------------	CARD 11 => CARD 10
	\draw[opacity=0,name path=segment1110] (card11) to[out=15, in=-80, looseness=2] (card10);
	\path[name intersections={of=card11 and segment1110, by={card1111}}];
	\node[opacity=0,fill=black, circle, minimum size=5pt, inner sep=0pt] at (card1111) {};
	\path[name intersections={of=card10 and segment1110, by={card1010}}];
	\node[opacity=0,fill=black, circle, minimum size=5pt, inner sep=0pt] at (card1010) {};
	\draw[line width = \lineWidthMore] decorate[decoration={markings, mark=between positions 0.3 and .9 step 11mm with {\arrow[scale=1.3]{stealth}}}] {(card1111.center) to[out=15, in=-80, looseness=2] (card1010.center)};
	\draw[line width = \lineWidthMore] (card1111.center) to[out=15, in=-80, looseness=2] (card1010.center);


%======================================================================














%======================================================================
% -----------------------------	definizione FRECCE inferiori ---------------------------------------------------
%
% ----------------------------	CARD 8 => CARD 6
    \gettikzxy{(card6)}{\bx}{\by}
    \pgfmathsetmacro{\angle}{360/(\numCards - 3) * (7) + \angleOffset}
    \def\fakeCircle{\circleRadius+\cardRadiusTwo+\cardRadiusTwo}
    \coordinate (car7away) at ($(\paperWidth/2,\paperHeight/2) + (\angle:\fakeCircle)$);
	\node[opacity=0,draw,fill=black,circle, minimum size=5pt, inner sep=0pt] (card7away) at (car7away) {};	
	\draw[opacity=0,name path=segment87away] (card8) to[out=0, in=90, looseness=0] (card7away);

	\draw[opacity=0,name path=segment67away] (card7away) to[out=270, in=0, looseness=0] (card6);
	\path[name intersections={of=card8 and segment87away, by={card88away}}];
	\node[opacity=0,fill=black, circle, minimum size=5pt, inner sep=0pt] at (card88away) {};
% calcolo l'angolo del punto
    \pgfmathanglebetweenpoints{\pgfpointanchor{card8}{center}}{\pgfpointanchor{card88away}{center}}
    \let\angle\pgfmathresult 
	\path[name intersections={of=card6 and segment67away, by={card66away}}];
	\node[opacity=0,fill=black, circle, minimum size=5pt, inner sep=0pt] at (card66away) {};
% calcolo l'angolo del punto
    \pgfmathanglebetweenpoints{\pgfpointanchor{card6}{center}}{\pgfpointanchor{card66away}{center}}
    \let\angleTwo\pgfmathresult 

	\draw[line width = \lineWidthMore] decorate[decoration={markings, mark=between positions 0.2 and .9 step 11mm with {\arrow[scale=1.3]{stealth}}}] {(card88away.center) to[out=\angle, in=\angle-230, in looseness=1] (card7away.center) to[out=\angleTwo+230, in=\angleTwo, out looseness=1] (card66away)};
	\draw[line width = \lineWidthMore] (card88away.center) to[out=\angle, in=\angle-230, in looseness=1] (card7away.center) to[out=\angleTwo+230, in=\angleTwo, out looseness=1] (card66away);













%======================================================================
% -----------------------------	definizione Licence ----------------------------------------------------------------
% 
%\node at (38,1) {\includegraphics[width=.8cm]{CC-license/cc.pdf}};  % Logo CC
%\node at (39,1) {\includegraphics[width=.8cm]{CC-license/by.pdf}};  % Logo BY
%\node at (40,1) {\includegraphics[width=.8cm]{CC-license/nc.pdf}};  % Logo NC
%\node at (41,1) {\includegraphics[width=.8cm]{CC-license/nd.pdf}};  % Logo ND
% ----------------------------	
% Aggiungi un link ipertestuale nella tikzpicture
%\node[left] at (41.5,3.5) {\href{https://github.com/DMGiulioRomano/strappi}{Jolt}};
%\node[left] at (41.5,2.75) {\href{https://github.com/DMGiulioRomano}{Giulio Romano De Mattia}};
%\node[left] at (41.5,2) {\href{https://creativecommons.org/licenses/by-nc-nd/4.0/?ref=chooser-v1}{Licensed under CC BY-NC-ND}};
%======================================================================




%\node [draw, ellipse,fit=(card4) (card5),inner sep=35pt, dashed,rotate fit=-30] {};
%\node [draw, ellipse,fit=(card6) (card7) (card8),inner sep=22pt, dashed,rotate fit=-15] {};
%\node [draw, ellipse,fit=(card10) (card11) (card9),inner sep=5pt, dashed,rotate fit=30] {};



\end{tikzpicture}



\end{document}
