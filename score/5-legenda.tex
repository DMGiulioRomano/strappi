\documentclass[tikz, a3paper, landscape, border=0]{standalone}

\usepackage{newpxtext}
\usepackage{newpxmath}
\usepackage{fontspec}
\usepackage{hyperref}  
 \usepackage[utf8]{inputenc}  
 \usepackage{ragged2e}  % Pacchetto per giustificare il testo

\catcode`\|12
% -----------------------------
\usetikzlibrary{calc}
\usetikzlibrary{intersections}
% -----------------------------
\usepackage{comandi}
% -----------------------------
\usetikzlibrary{decorations.markings}
\usetikzlibrary {decorations.text}
\usetikzlibrary { decorations.pathmorphing, decorations.pathreplacing, decorations.shapes}
\usetikzlibrary{matrix}
\usetikzlibrary {quotes} 
  
 % Definisci una macro per il testo
\newcommand{\testoPreliminare}{ Il brano è concepito come un processo di scoperta dello strumento da parte del percussionista, il quale, attraverso un'esplorazione controllata, individua le possibili micro-variazioni timbriche prodotte. Tali variazioni risultano significative in relazione ai gesti eseguiti precedentemente e a quelli che seguiranno, creando una continuità e un'interdipendenza tra le azioni sonore, dove ogni gesto influenza e viene influenzato dalle perturbazioni timbriche circostanti.}
\newcommand{\testoSemiografia}{Le seguenti griglie spiegano come interpretare i gesti scomponendo i vari \mbox{elementi} in oggetti primitivi a partire dalla distinzione dei simboli di:}
\newcommand{\Modalitadimpugnatura}{La modalità \textit{Ordinaria/Capovolta} si riferisce alla campana con la parte concava del corpo rivolta rispettivamente verso l'alto/il basso. La tipologia \textit{Libera/Bloccata} si riferisce all'impugnatura della mano che permette alla campana di vibrare o di essere letteralmente bloccata in modo da non permettere una risonanza.}
\newcommand{\Puntodipercussionedelbattente}{La tipologia Libera/Bloccata in questo caso si riferisce alla superficie del battente che entra in contatto con il corpo della campana ed è rispettivamente legno/feltro.}




\newcommand{\testoSemiografiaTwo}{La relazione tra questi due primitivi può essere osservata nella modalità di attacco in relazione allo spazio della campana, e in funzione dell'azione che svolge il battente.}
\newcommand{\Puntidipercussionedellacampanainrelazionealbattente}{Questi sono i tre gesti che idealmente permettono di eccitare in maniera differente i modi di vibrazione della campana nonostante questi siano difficilmente controllabili data la modalità artigianale di fabbricazione dello strumento.}
\newcommand{\UlterioriGesti}{\textit{Strofinato}: movimento ciclico attorno la campana per generare un suono continuo ma mobile variando la quantità di pressione e di velocità. \textit{Trattenuto}: percussione mantenendo il battente attaccato alla campana (senza il rimbalzo necessario per la risonanza).}





\newcommand{\testoAcciaccaturaCampana}{L'articolazione nel tempo di queste due tipologie è definito come \textit{Acciaccatura} ed è rappresentata con un passaggio di stato (da Libera a Bloccata e viceversa); il gesto è condensato in un unico simbolo e va letto da sinistra verso destra.}

\newcommand{\acciaccaturaCampana}{L'acciaccatura è da intendersi come smorzamento della risonanza della campana nel tempo. La campana permette uno smorzamento graduale, mentre il battente prevede una discontinuità nell'alternanza tra legno e feltro. Il tempo di smorzamento è variabile in base alle relazioni tra i gesti.}


\newcommand{\testoAcciaccaturaBattente}{L'articolazione nel tempo di queste due tipologie è definito come \textit{Acciaccatura} ed è rappresentata con un passaggio di stato (da Legno a Feltro e viceversa); il gesto è condensato in un unico simbolo e va letto da sinistra verso destra.}

\newcommand{\acciaccaturaBattente}{L'acciaccatura è da intendersi come smorzamento della risonanza della campana nel tempo. Il battente prevede una discontinuità nell'alternanza tra legno e feltro. Il tempo di smorzamento è variabile in base alle relazioni tra i gesti e l'acciaccatura può essere sul silenzio.}

\newcommand{\void}{* \textit{Void} si riferisce a tutte le possibili condizioni correlate non specificate poiché trascurabili. \' E interpretabile come \textit{neutro}.}



\begin{document}
\begin{tikzpicture}[every edge quotes/.style={fill=white,font=\footnotesize}]


%======================================================================
% ------------------------------ preparazione foglio di lavoro ----------------------------------------------------
\def\paperHeight{29.7cm}
\def\paperWidth{42cm}
\fill[white] (0,0) rectangle ++ (\paperWidth,\paperHeight);
%\draw[step=.01cm, gray, line width = 0.01pt] (0, 0) grid (\paperWidth, \paperHeight); 
%\draw[step=.1cm, gray, very thin] (0, 0) grid (\paperWidth, \paperHeight); 
%\draw[step=.5cm, gray, ultra thin] (0, 0) grid (\paperWidth, \paperHeight);
%\draw[step=1cm, black, very thin] (0, 0) grid (\paperWidth, \paperHeight); 
%\draw[black, very thick, dotted] (\paperWidth/2,\paperHeight) -- (\paperWidth/2,0);
%======================================================================



%======================================================================
% Posizionamento delle carte in cerchio
\def\circleRadius{11cm}  % Regola questo valore in base alle tue esigenze
\def\numCards{11}  % Numero di carte nel cerchio principale
\def\angleOffset{30}  % Offset angolare per ruotare l'intero layout
% ------------------
\def\pointWidth{3pt}
% ------------------
\def\minimumSize{1.5pt}
% ------------------
\def\lineWidth{1.1pt}
\def\arrowWidth{6pt}
\def\arrowLength{9pt}
\def\sumWidth{1pt}
\def\lineWidthMore{2.5pt}
\def\arrowWidthMore{9pt}
\def\arrowLengthMore{12pt}
\def\sumWidthMore{3pt}
\def\roundedCorners{8pt}
% ------------------
\def\cardRadius{2.1cm} 
\def\cardRadiusTwo{\cardRadius + .5cm} 
% ------------------
\def\battenteheight{.25}
\def\battentewidth{1.3}
%======================================================================












%======================================================================
% ------------------------------ Tabella Campana  ------------------------------------------------------------------
%
 \matrix (tableCampana) [
    matrix of nodes,
    nodes={align=center, text width=2.75cm, minimum height=1.75cm, inner sep = 0pt},
    row sep=-\pgflinewidth,
    column sep=-\pgflinewidth,
    nodes in empty cells
] at (\paperWidth/6,\paperHeight/1.4) {
    |[font=\bfseries]| & |[font=\bfseries]|  Libera & |[font=\bfseries]|  Bloccata\\
    |[font=\bfseries]| Ordinaria &  &   \\
    |[font=\bfseries]| Capovolta  &  &   \\
    };

\node[   align=justify, text width=8cm, minimum height=.75cm,  inner sep = 0pt, below=.8cm] at (tableCampana-3-2) {\justify \Modalitadimpugnatura};
% Disegna i bordi esterni
\draw [rounded corners = 2pt](tableCampana-1-1.north west) rectangle (tableCampana-3-3.south east);

\campana{0}{0.05}{.8}{ordinaria}{libera}{tableCampana-2-2}
\campana{0}{0.05}{.8}{ordinaria}{bloccata}{tableCampana-2-3}
\campana{0}{.1}{.8}{capovolta}{libera}{tableCampana-3-2}
\campana{0}{.1}{.8}{capovolta}{bloccata}{tableCampana-3-3}
%======================================================================




%======================================================================
% ------------------------------ Tabella Battente  --------------------------------------------------------------------
%
 \matrix (tableBattente) [
    matrix of nodes,
    nodes={align=center, text width=2.75cm, minimum height=1.75cm, inner sep = 0pt},
    row sep=-\pgflinewidth,
    column sep=-\pgflinewidth,
    nodes in empty cells
] at ($(tableCampana) + (\paperWidth/1.5, 1.035cm)$) {
    |[font=\bfseries]| & |[font=\bfseries]|  Libera & |[font=\bfseries]|  Bloccata\\
    |[font=\bfseries]| Void *&  &   \\
    };
\node[   align=justify, text width=8cm, minimum height=.75cm,  inner sep = 0pt, below=1.2cm] at (tableBattente-2-2) {\Puntodipercussionedelbattente};

% Disegna i bordi esterni
\draw[rounded corners = 2pt] (tableBattente-1-1.north west) rectangle (tableBattente-2-3.south east);
\updateBattentePosition{0}{0.1}
\battente{tableBattente-2-2}{\battenteStartX}{\battenteStartY}{\battentewidth}{\battenteheight}{libero}{0}
\battente{tableBattente-2-3}{\battenteStartX}{\battenteStartY}{\battentewidth}{\battenteheight}{bloccato}{0}
%======================================================================








%======================================================================
% ------------------------------ Tabella Acciaccatura Campana -------------------------------------------------
%
\matrix (campanaAcciaccatura) [
    matrix of nodes,
    nodes={align=center, text width=2.75cm, minimum height=1.75cm, inner sep = 0pt},
    row sep=-\pgflinewidth,
    column sep=-\pgflinewidth,
    nodes in empty cells
] at (\paperWidth/6,\paperHeight/3.65) {
    |[font=\bfseries]| & |[font=\bfseries]| & |[font=\bfseries]|  \\
    |[font=\bfseries]|  & |[font=\bfseries]| &   \\
    };
    \node[align=justify, text width=8cm, minimum height=.75cm,  inner sep = 0pt, below=1.2cm] at (campanaAcciaccatura-2-2) {\acciaccaturaCampana};
    
% Disegna i bordi esterni
\draw [rounded corners = 2pt] (campanaAcciaccatura-1-1.north west) rectangle (campanaAcciaccatura-2-3.south east);

\campanaAccZ{0}{0.05}{.8}{libera}{campanaAcciaccatura-1-1}
\campana{0}{0.05}{.8}{ordinaria}{libera}{campanaAcciaccatura-1-2}
\campana{0}{0.05}{.8}{ordinaria}{bloccata}{campanaAcciaccatura-1-3}
\campanaAccZ{0.015}{0.05}{.8}{bloccata}{campanaAcciaccatura-2-1}
\campana{0}{0.05}{.8}{ordinaria}{bloccata}{campanaAcciaccatura-2-3}
\campana{0}{0.05}{.8}{ordinaria}{libera}{campanaAcciaccatura-2-2}
\draw (campanaAcciaccatura-1-1.east) edge ["\large \bfseries=", ->, out=180, in=60, looseness=.5] (campanaAcciaccatura-1-2.west);
\draw (campanaAcciaccatura-2-1.east) edge ["\large \bfseries=", ->, out=180, in=60, looseness=.5] (campanaAcciaccatura-2-2.west);
\draw ($(campanaAcciaccatura-1-2.center)+(1,0)$) edge [ thick,->, out=0, in=180, looseness=.5] ($(campanaAcciaccatura-1-3.center)-(1,0)$);
\draw ($(campanaAcciaccatura-2-2.center)+(1,0)$) edge [ thick,->, out=0, in=180, looseness=.5] ($(campanaAcciaccatura-2-3.center)-(1,0)$);
%======================================================================




%======================================================================
% ------------------------------ Tabella Acciaccatura Battente ---------------------------------------------------
%
\matrix (battenteAcciaccatura) [
    matrix of nodes,
    nodes={align=center, text width=2.75cm, minimum height=1.75cm, inner sep = 0pt},
    row sep=-\pgflinewidth,
    column sep=-\pgflinewidth,
    nodes in empty cells
] at (\paperWidth-\paperWidth/6,\paperHeight/2.9) {
    |[font=\bfseries]| & |[font=\bfseries]|  & |[font=\bfseries]| \\
    |[font=\bfseries]|  &  &   \\
    };
    \node[align=justify, text width=8cm, minimum height=.75cm,  inner sep = 0pt, below=1.2cm] at (battenteAcciaccatura-2-2) {\acciaccaturaBattente};
    
% Disegna i bordi esterni
\draw [rounded corners = 2pt] (battenteAcciaccatura-1-1.north west) rectangle (battenteAcciaccatura-2-3.south east);


\updateBattenteAccPosition{0.005}{0}
\battenteAcc{battenteAcciaccatura-1-1}{\battenteStartAccX}{\battenteStartAccY}{\battentewidth}{\battenteheight}{aperto}{0}
\updateBattentePosition{0}{0}
\battente{battenteAcciaccatura-1-2}{\battenteStartX}{\battenteStartY}{\battentewidth}{\battenteheight}{libero}{0}
\battente{battenteAcciaccatura-1-3}{\battenteStartX}{\battenteStartY}{\battentewidth}{\battenteheight}{bloccato}{0}


\updateBattenteAccPosition{0.015}{0}
\battenteAcc{battenteAcciaccatura-2-1}{\battenteStartAccX}{\battenteStartAccY}{\battentewidth}{\battenteheight}{bloccato}{0}
\battente{battenteAcciaccatura-2-2}{\battenteStartX}{\battenteStartY}{\battentewidth}{\battenteheight}{bloccato}{0}
\battente{battenteAcciaccatura-2-3}{\battenteStartX}{\battenteStartY}{\battentewidth}{\battenteheight}{libero}{0}

\draw (battenteAcciaccatura-1-1.east) edge ["\large \bfseries=", ->, out=180, in=60, looseness=.5] (battenteAcciaccatura-1-2.west);
\draw (battenteAcciaccatura-2-1.east) edge ["\large \bfseries=", ->, out=180, in=60, looseness=.5] (battenteAcciaccatura-2-2.west);
\draw ($(battenteAcciaccatura-1-2.center)+(1,0)$) edge [ thick,->, out=0, in=180, looseness=.5] ($(battenteAcciaccatura-1-3.center)-(1,0)$);
\draw ($(battenteAcciaccatura-2-2.center)+(1,0)$) edge [ thick,->, out=0, in=180, looseness=.5] ($(battenteAcciaccatura-2-3.center)-(1,0)$);
%======================================================================









%======================================================================
% ------------------------------ Tabella Campana - battente  -----------------------------------------------------
%
 \matrix (tableBattenteCampana) [
    matrix of nodes,
    nodes={align=center, text width=2.4cm, minimum height=2.2cm, inner sep = 0pt},
    row sep=-\pgflinewidth,
    column sep=-\pgflinewidth,
    nodes in empty cells
] at (\paperWidth/1.7,\paperHeight/1.77) {
    |[font=\bfseries]| & |[font=\bfseries]|  Void *\\
    |[font=\bfseries]| Dall'alto & \\
    |[font=\bfseries]| Sulla pancia  & \\
    |[font=\bfseries]| De dentro  & \\
    };

\gettikzxy{(tableBattenteCampana-4-2)}{\cbx}{\cby}
  
\node[align=justify,text width=4.8cm, minimum height=1cm,  inner sep = 0pt, below=1cm] at (\cbx-1.2cm,\cby-.5cm) {\Puntidipercussionedellacampanainrelazionealbattente};

% Disegna i bordi esterni
\draw [rounded corners = 2pt] (tableBattenteCampana-1-1.north west) rectangle (tableBattenteCampana-4-2.south east);

\campana{0}{-.3}{.8}{ordinaria}{libera}{tableBattenteCampana-2-2}
\campana{0}{-.3}{.8}{ordinaria}{libera}{tableBattenteCampana-3-2}
\campana{0}{-.3}{.8}{ordinaria}{libera}{tableBattenteCampana-4-2}

\updateBattentePosition{-.35}{.02}
\battente{tableBattenteCampana-2-2}{\battenteStartX}{\battenteStartY}{\battentewidth}{\battenteheight}{libero}{45}
\updateBattentePosition{-.45}{-.3}
\battente{tableBattenteCampana-3-2}{\battenteStartX}{\battenteStartY}{\battentewidth}{\battenteheight}{libero}{0}
\updateBattentePosition{0.36}{-.4}
\battente{tableBattenteCampana-4-2}{\battenteStartX}{\battenteStartY}{\battentewidth}{\battenteheight}{libero}{80}
\draw[line width = .8pt] ($(tableBattenteCampana-4-2)+(-.7,+.1)$) -- ($(tableBattenteCampana-4-2)+(+.7,+.1)$);

%======================================================================











%======================================================================
% ------------------------------ Tabella Interazioni primitive  -----------------------------------------------------
%
 \matrix (tableTipologie) [
    matrix of nodes,
    nodes={align=center, text width=2.4cm, minimum height=2.2cm, inner sep = 0pt},
    row sep=-\pgflinewidth,
    column sep=-\pgflinewidth,
    nodes in empty cells
] at (\paperWidth/2.4,\paperHeight/1.9) {
    |[font=\bfseries]| & |[font=\bfseries]|  Void *\\
    |[font=\bfseries]| Impulsivo & \\
    |[font=\bfseries]| Strofinato & \\
    |[font=\bfseries]| Trattenuto  & \\
    |[font=\bfseries]| Ribattuto  & \\
    };

\gettikzxy{(tableTipologie-5-2)}{\cbx}{\cby}
  
\node[align=justify,text width=4.8cm, minimum height=1cm,  inner sep = 0pt, below=1cm] at (\cbx-1.2cm,\cby) {\justify \UlterioriGesti};

% Disegna i bordi esterni
\draw [rounded corners = 2pt] (tableTipologie-1-1.north west) rectangle (tableTipologie-5-2.south east);

\campana{0}{-.3}{.8}{ordinaria}{libera}{tableTipologie-2-2}
\campana{0}{-.3}{.8}{ordinaria}{libera}{tableTipologie-3-2}
\campana{0}{-.3}{.8}{ordinaria}{libera}{tableTipologie-4-2}
\campana{0}{-.3}{.8}{ordinaria}{libera}{tableTipologie-5-2}

\updateBattentePosition{0}{.02}
\battente{tableTipologie-2-2}{\battenteStartX}{\battenteStartY}{\battentewidth}{\battenteheight}{libero}{60}

\updateBattentePosition{0}{.02}
\battente{tableTipologie-3-2}{\battenteStartX}{\battenteStartY}{\battentewidth}{\battenteheight}{libero}{60}
\strofinato{tableTipologie-3-2}{.8}{.1}{-1.45}{.4}

\updateBattentePosition{0}{.23}
\battente{tableTipologie-4-2}{\battenteStartX}{\battenteStartY}{\battentewidth}{\battenteheight}{aperto}{0}
\draw[arrows = {-Latex}, double distance=1.2pt] ($(tableTipologie-4-2) + (0,1.2)$) -- ($(tableTipologie-4-2) + (0,.33)$);

\updateBattentePosition{-.45}{-.3}
\battente{tableTipologie-5-2}{\battenteStartX}{\battenteStartY}{\battentewidth}{\battenteheight}{libero}{0}
\draw[line width = .6] ($(tableTipologie-5-2) + (-.2, .3)$) --  ++(.4,.2);
\draw[line width = .6] ($(tableTipologie-5-2) + (-.2, .4)$) --  ++(.4,.2);
\draw[line width = .6] ($(tableTipologie-5-2) + (-.2, .5)$) --  ++(.4,.2);
%======================================================================







%======================================================================
% ------------------------------  Nodi testuali   -----------------------------------------------------------------------
%
    \node[align=justify,below, text width=\paperWidth-18.5cm] at (\paperWidth/2,\paperHeight-1cm) {\justify \testoPreliminare };
    \node[align=center,minimum height=1cm,  below, text width=\paperWidth/2-5.5cm, inner sep = 0pt] (testoSemiografia) at (\paperWidth/2,\paperHeight-3cm) {\justify \testoSemiografia};
    \node[align=center,minimum height=1cm,  below, text width=\paperWidth/2-5.5cm, inner sep = 0pt] (testoSemiografiaDue) at (\paperWidth/2,\paperHeight-6.5cm) {\justify \testoSemiografiaTwo};
    \node[minimum height=1.5cm, align=justify, below, text width=8cm] (testoAcciaccaturaCampana) at (\paperWidth/6,\paperHeight/2.2) { \justify \testoAcciaccaturaCampana};
    \node[minimum height=1.5cm, align=justify, below, text width=8cm] (testoAcciaccaturaBattente) at (\paperWidth-\paperWidth/6,\paperHeight/1.9) { \justify \testoAcciaccaturaBattente};
%
% ------------
    \node[minimum height=1.5cm, align=center, below, text width=\paperWidth/3-1cm] (void) at (\paperWidth/2,	\paperHeight/9) {\justify \void};


%======================================================================











%======================================================================
% ------------------------------  Frecce  --------------------------------------------------------------------------------
%
\draw (testoSemiografia.west) edge ["Campana", ->, out=180, in=60, looseness=.5] (tableCampana-1-2.north);
\draw (testoSemiografia.east) edge ["Battente", ->, out=0, in=120, looseness=.5] (tableBattente-1-2.north);
\draw (tableBattente-2-1.west) edge [ ->, out=180, in=180, looseness=1] (testoAcciaccaturaBattente.west);
\draw (tableCampana-2-3.east) edge [ ->, out=0, in=0, looseness=1] (testoAcciaccaturaCampana.east);
%\draw (RightTestoSemiografia) edge ["Battente", ->] (tableBattente-1-1.north);
%\node (newTableAcciaccaturaWest) at (tableAcciaccatura-2-1.west) [above=.2] {};
%\node (newTableAcciaccaturaEast) at (tableAcciaccatura-2-3.east) [above=.2] {};
%\draw (tableCampana-2-3.east) edge ["Acciaccatura", ->] (newTableAcciaccaturaWest);
%\draw (tableBattente-2-1.west) edge ["Acciaccatura", ->] (newTableAcciaccaturaEast);

%======================================================================






%======================================================================
% -----------------------------	definizione Licence ----------------------------------------------------------------
% 
\node at (38,1) {\includegraphics[width=.8cm]{CC-license/cc.pdf}};  % Logo CC
\node at (39,1) {\includegraphics[width=.8cm]{CC-license/by.pdf}};  % Logo BY
\node at (40,1) {\includegraphics[width=.8cm]{CC-license/nc.pdf}};  % Logo NC
\node at (41,1) {\includegraphics[width=.8cm]{CC-license/nd.pdf}};  % Logo ND
% ----------------------------	
% Aggiungi un link ipertestuale nella tikzpicture
\node[left] at (41.5,3.5) {\href{https://github.com/DMGiulioRomano/strappi}{Jolt}};
\node[left] at (41.5,2.75) {\href{https://github.com/DMGiulioRomano}{Giulio Romano De Mattia}};
\node[left] at (41.5,2) {\href{https://creativecommons.org/licenses/by-nc-nd/4.0/?ref=chooser-v1}{Licensed under CC BY-NC-ND}};
%======================================================================


\end{tikzpicture}
\end{document}



