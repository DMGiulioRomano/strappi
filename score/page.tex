\documentclass[tikz, a3paper, landscape, border=0]{standalone}
\usetikzlibrary{calc}
\usepackage{comandi}
\usepackage{musicography}

\begin{document}
\begin{tikzpicture}

%======================================================================
% ------------------------------ funzione per ottenere x e y di un nodo ----------------------------------------
\makeatletter
\newcommand{\gettikzxy}[3]{%
  \tikz@scan@one@point\pgfutil@firstofone#1\relax
  \edef#2{\the\pgf@x}%
  \edef#3{\the\pgf@y}%
}
\makeatother
%======================================================================







%======================================================================
% ------------------------------ preparazione foglio di lavoro ----------------------------------------------------
\def\paperHeight{29.7}
\def\paperWidth{42}
\fill[white] (0,0) rectangle ++ (\paperWidth,\paperHeight);
% Imposta la griglia del foglio A3 come riferimento (42x29.7 cm)
%\draw[step=.1cm, gray, very thin] (0, 0) grid (\paperWidth, \paperHeight); % Griglia per visualizzare lo spazio A3
%\draw[step=1cm, black, very thin] (0, 0) grid (\paperWidth, \paperHeight); % Griglia per visualizzare lo spazio A3
%======================================================================



%======================================================================


%======================================================================









%======================================================================
% ------------------------------ definizione variabili per sezione centrale del brano -----------------------
\def\cardHeight{5} 
\def\cardwidth{3} 
\def\cardSpacing{1.5}
\def\cardOffset{-.75}
\def\cardY{\paperHeight/2-\cardHeight/2}
% ------------------
\def\battenteheight{.25}
\def\battentewidth{1.3}
% ------------------
\def\pointWidth{2.3pt}
\def\lineWidth{1.3pt}
% ------------------
\def\arrowWidth{7pt}
\def\arrowLength{10pt}
%======================================================================






%======================================================================
% -----------------------------	carte 	INTRODUZIONE	------------------------------------------------
%
\foreach \i  in {1,...,3}{
        		\edef\cardname{card\i}
        		\edef\cardin{in\i}
        		\edef\cardout{out\i}
        		\card{-3+\cardSpacing*\cardwidth*\i}{\cardY+10}{\cardwidth}{\cardHeight}{\cardname}{\cardin}{\cardout}
}
%======================================================================








%====================================================================
% -----------------------------	carte 	CENTRALI	----------------------------------------------------
%
    	\foreach \i [evaluate=\i as \cont using int(\i-3)] in {4,...,11} {
        % Genera i rettangoli con nomi incrementali card1, card2, etc.
        		\edef\cardname{card\i}
        		\edef\cardin{in\i}
        		\edef\cardout{out\i}
        		\card{\cardOffset+\cardSpacing*\cardwidth*\cont}{\cardY}{\cardwidth}{\cardHeight}{\cardname}{\cardin}{\cardout}
    	}
\card{\cardOffset+\cardSpacing*\cardwidth*3}{\cardY-\cardHeight-2.5}{\cardwidth}{\cardHeight}{card6bis}{in6bis}{out6bis}
%====================================================================







%====================================================================
% -----------------------------	carte 	FINALE		----------------------------------------------------
%
\card{24+\cardSpacing*\cardwidth*3}{\cardY-10}{\cardwidth}{\cardHeight}{card12}{in12}{noout}
%====================================================================











%======================================================================
% -----------------------------	definizione FRECCE carte introduzione -------------------------------
%	
	\foreach \i [evaluate=\i as \next using int(\i+1)] in {1,...,2}{
        		\edef\cardin{in\next}
        		\edef\cardout{out\i}
		\edef\a{a\i}
		\edef\b{b\i}
		\edef\c{c\i}	
		\draw[arrows = {-Kite[black, width=\arrowWidth, length=\arrowLength, line width = \lineWidth]},line width = \lineWidth]  (\cardout.center) -- (\cardin.center);
	}
%======================================================================
















%======================================================================
% -----------------------------	definizione NODI card 3  ---------------------------------------------------------
%	
    % Definizione di punti di riferimento
    \coordinate (Z) at ($(out3.center) + (1,0)$);
    \coordinate (A) at ($(Z) - (0,\cardHeight/1.2)$);
    \coordinate (B) at ($(A) - (\cardwidth*4.2,0)$);
    \gettikzxy{(B)}{\bx}{\by}
    \gettikzxy{(in4)}{\inx}{\iny}
    \coordinate (C) at (\bx,\iny);
    \draw[arrows = {-Kite[black, width=\arrowWidth, length=\arrowLength, line width = \lineWidth]},line width = \lineWidth, rounded corners=10pt] 	(out3.center) -- (Z) -- (A)-- (B) -- (C) -- node[pos=.3, name=a3] {} node[pos=0.55, name=b3] {} node[pos=0.7, name=c3] {}(in4.center);
%======================================================================










%======================================================================
% -----------------------------	definizione FRECCE e NODI carte -------------------------------------------
%
% ----------------------------	CARD	4
\def\i{4}
\def\nexti{5}
\draw[arrows = {-Kite[black, width=\arrowWidth, length=\arrowLength, line width = \lineWidth]},line width = \lineWidth, rounded corners=10pt]  (out\i.center) --  (in\nexti.center)
	node[pos =0.3,circle, draw=black, inner sep=0mm, line width = \pointWidth] (a\i) {};
% ----------------------------	CARD	5
\def\i{5}
\def\nexti{6}
\draw[arrows = {-Kite[black, width=\arrowWidth, length=\arrowLength, line width = \lineWidth]},line width = \lineWidth, rounded corners=10pt]  (out\i.center) -- (in\nexti.center)
	node[pos =0.2,circle, draw, fill=black, inner sep=0mm, line width = \pointWidth] (a\i) {}
        	node[pos=.45, inner sep=0mm, line width = \pointWidth] (b\i) {}
        	node[pos=.65, inner sep=0mm, line width = \pointWidth] (c\i) {};
% ----------------------------	CARD	6
\def\i{6}
\def\nexti{7}
\draw[arrows = {-Kite[black, width=\arrowWidth, length=\arrowLength, line width = \lineWidth]},line width = \lineWidth]  (out\i.center) -- (in\nexti.center)
	node[pos =0.2,circle, draw, fill=black, inner sep=0mm, line width = \pointWidth] (a\i) {}
        	node[pos=.4,inner sep=0mm, line width = \pointWidth] (b\i) {}
        	node[pos=.6, inner sep=0mm, line width = \pointWidth] (c\i) {};
% ----------------------------	CARD	7
\def\i{7}
\def\nexti{8}
\draw[arrows = {-Kite[black, width=\arrowWidth, length=\arrowLength, line width = \lineWidth]},line width = \lineWidth]  (out\i.center) -- (in\nexti.center);
% ----------------------------	CARD	8
\def\i{8}
\def\nexti{9}
\draw[arrows = {-Kite[black, width=\arrowWidth, length=\arrowLength, line width = \lineWidth]},line width = \lineWidth]  (out\i.center) -- (in\nexti.center)
	node[pos =0.2,circle, draw, fill=black, inner sep=0mm, line width = \pointWidth] (a\i) {}
        	node[pos=.4,circle, draw, fill=black, inner sep=0mm, line width = \pointWidth] (b\i) {}
        	node[pos=.7,inner sep=0mm, line width = \pointWidth] (c\i) {};
% ----------------------------	CARD	9
\def\i{9}
\def\nexti{10}
\draw[arrows = {-Kite[black, width=\arrowWidth, length=\arrowLength, line width = \lineWidth]},line width = \lineWidth]  (out\i.center) -- (in\nexti.center)
	node[pos =0.2,circle, draw, fill=black, inner sep=0mm, line width = \pointWidth] (a\i) {}
        	node[pos=.6,inner sep=0mm, line width = \pointWidth] (b\i) {};
% ----------------------------	CARD	10
\def\i{10}
\def\nexti{11}
\draw[arrows = {-Kite[black, width=\arrowWidth, length=\arrowLength, line width = \lineWidth]},line width = \lineWidth]  (out\i.center) -- (in\nexti.center);
% ----------------------------	CARD	11
\gettikzxy{(out11)}{\outx}{\outy}
\gettikzxy{(in11)}{\inx}{\iny}
\gettikzxy{(card12)}{\cardx}{\cardy}
\gettikzxy{(card10)}{\cardtenx}{\cardteny}

\coordinate (A) at (\cardx + 1cm,\outy);
\coordinate (B) at ($(A) - (0,\cardHeight*1.2)$);
\gettikzxy{(B)}{\bx}{\by}
\coordinate (C) at (\cardtenx,\by);
\coordinate (D) at (\cardtenx,\cardy);

\draw[arrows = {-Kite[black, width=\arrowWidth, length=\arrowLength, line width = \lineWidth]},line width = \lineWidth,  rounded corners=10pt] 	(out11.center) -- node[pos =0.3,circle, draw, fill=black, inner sep=0mm, line width = \pointWidth] (a11) {} (A) -- (B) -- (C) -- node[pos =0.8,circle, draw, fill=black, inner sep=0mm, line width = \pointWidth] (b11) {} (D) -- (in12.center);
%======================================================================













%======================================================================
% -----------------------------	definizione campane		-------------------------------------------------------
%
\campana{0}{0}{.8}{capovolta}{libera}{card1}
\campana{0}{0}{.8}{capovolta}{libera}{card2}
\campana{0}{0}{.8}{ordinaria}{bloccata}{card3}
\campana{0}{0}{.8}{ordinaria}{bloccata}{card4}
\campana{0}{0}{.8}{ordinaria}{libera}{card5}
\campana{0}{0}{.8}{ordinaria}{libera}{card6}
\campanaAcc{0}{0}{.8}{bloccata}{card6bis}
\campanaAcc{0}{0}{.8}{libera}{card7}
\campanaAcc{0}{0}{.8}{libera}{card8}
\campanaAcc{0}{0}{.8}{libera}{card9}
\campanaAcc{0}{0}{.8}{libera}{card10}
\campanaAcc{0}{0}{.8}{libera}{card11}
\campana{0}{0}{.8}{ordinaria}{bloccata}{card12}
%======================================================================











%======================================================================
% -----------------------------	definizione battenti		-------------------------------------------------------
%
% ----------------------------	CARD	1
\updateBattentePosition{0.1}{.3}
\battente{card1}{\battenteStartX}{\battenteStartY}{\battentewidth}{\battenteheight}{libero}{60}
\strofinato{card1}{.65}{.1}{-1.15}{.7}
% ----------------------------	CARD	2
\updateBattentePosition{0}{.53}
\battente{card2}{\battenteStartX}{\battenteStartY}{\battentewidth}{\battenteheight}{libero}{0}
\draw[arrows = {-Latex}, double distance=1.2pt] ($(card2) + (0,1.6)$) -- ($(card2) + (0,.63)$);
% ----------------------------	CARD	3
\updateBattentePosition{0}{.32}
\battente{card3}{\battenteStartX}{\battenteStartY}{\battentewidth}{\battenteheight}{libero}{60}
\strofinato{card3}{.8}{.1}{-1.45}{.7}
% ----------------------------	CARD	4
\updateBattentePosition{-.35}{.32}
\battente{card4}{\battenteStartX}{\battenteStartY}{\battentewidth}{\battenteheight}{libero}{45}
% ----------------------------	CARD	5
\updateBattenteAccPosition{0}{.53}
\battenteAcc{card5}{\battenteStartAccX}{\battenteStartAccY}{\battentewidth}{\battenteheight}{aperto}{0}
\draw[arrows = {-Latex}, double distance=1.2pt] ($(card5) + (0,1.6)$) -- ($(card5) + (0,.63)$);
% ----------------------------	CARD	6
\updateBattentePosition{-.45}{.0}
\battente{card6}{\battenteStartX}{\battenteStartY}{\battentewidth}{\battenteheight}{bloccato}{0}
% ----------------------------	CARD	6bis
\updateBattentePosition{-.35}{.32}
\battente{card6bis}{\battenteStartX}{\battenteStartY}{\battentewidth}{\battenteheight}{bloccato}{45}
% ----------------------------	CARD	7
\updateBattentePosition{0}{.32}
\battente{card7}{\battenteStartX}{\battenteStartY}{\battentewidth}{\battenteheight}{bloccato}{60}
\strofinato{card7}{.8}{.1}{-1.45}{.7}
% ----------------------------	CARD	8
\updateBattentePosition{-.35}{.32}
\battente{card8}{\battenteStartX}{\battenteStartY}{\battentewidth}{\battenteheight}{bloccato}{45}
% ----------------------------	CARD	9
\updateBattentePosition{0}{.32}
\battente{card9}{\battenteStartX}{\battenteStartY}{\battentewidth}{\battenteheight}{libero}{60}
\strofinato{card9}{.8}{.1}{-1.45}{.7}
% ----------------------------	CARD	10
\updateBattentePosition{-.45}{.0}
\battente{card10}{\battenteStartX}{\battenteStartY}{\battentewidth}{\battenteheight}{libero}{0}
% ----------------------------	CARD	11
\updateBattentePosition{-.35}{.32}
\battente{card11}{\battenteStartX}{\battenteStartY}{\battentewidth}{\battenteheight}{bloccato}{45}
% ----------------------------	CARD	12
\updateBattentePosition{0.36}{-.1}
\battente{card12}{\battenteStartX}{\battenteStartY}{\battentewidth}{\battenteheight}{libero}{80}
\draw[line width = .8pt] ($(card12)+(-.8,+.4)$) -- ($(card12)+(+.8,+.4)$);
\draw[->]   ($(card12)+(-.5,-.4)$) to[out=-30, in=200, looseness=1] ($(card12)+(.6,-.4)$);
\campanaSbiadita{0}{0}{.8}{ordinaria}{bloccata}{card12}
%======================================================================









%======================================================================
% -----------------------------	definizione dinamiche -------------------------------------------------------------
\def\dinamicY{\cardHeight/2.7}
% ----------------------------	SEPARATORE
\newcommand{\separatore}[3]{
\draw[draw=black, line width = \lineWidth] ($(#1) - (0, #3)$) -- ($(#2) - (0,#3)$);
\coordinate (midPoint) at ($(#1) - (-\cardwidth/2, #3 + #3/4)$);
}
% ----------------------------	CARD	1
\separatore{in1}{out1}{\cardHeight/4}
    \node at ($(card1) - (0, \dinamicY)$) {\Large \textit{mf} -- \textit{ff}};
% ----------------------------	CARD	2
\separatore{in2}{out2}{\cardHeight/4}
    \node at ($(card2) - (0, \dinamicY)$) {\Large \textit{mf} };
% ----------------------------	CARD	3
\separatore{in3}{out3}{\cardHeight/4}
    \node at ($(card3) - (0, \dinamicY)$) {\Large \textit{mf} -- \textit{f}};
% ----------------------------	CARD	4
\separatore{in4}{out4}{\cardHeight/4}
    \node at ($(card4) - (0, \dinamicY)$) {\Large \textit{mp} -- \textit{mf}};
% ----------------------------	CARD	5
\separatore{in5}{out5}{\cardHeight/4}
    \node at ($(card5) - (0, \dinamicY)$) {\Large \textit{pp} -- \textit{mf}};
% ----------------------------	CARD	6
\separatore{in6}{out6}{\cardHeight/4}
\coordinate (A) at ($(card6) - (\cardwidth/4, \dinamicY)$);
\coordinate (B) at ($(A) + (\cardwidth/2.5, \cardHeight/25)$);
\coordinate (C) at ($(A) + (\cardwidth/2.5, -\cardHeight/25)$);
\coordinate (D) at ($(card6) - (-\cardwidth/4,\dinamicY)$);
\node[draw=black, circle,  line width = .6pt, inner sep=1pt] at (A) {};
    \draw[thick] ($(A)+(.04,0)$) -- (B);  % Linea superiore del crescendo
    \draw[thick] ($(A)+(.04,0)$) -- (C); % Linea inferiore del crescendo
	\node at (D) {\Large \textit{f}};
% ----------------------------	CARD	6bis
\coordinate (A) at ($(card6bis) - (\cardwidth/4, \dinamicY)$);
\coordinate (B) at ($(A) + (\cardwidth/2.5, \cardHeight/25)$);
\coordinate (C) at ($(A) + (\cardwidth/2.5, -\cardHeight/25)$);
\coordinate (D) at ($(card6bis) - (-\cardwidth/4,\dinamicY)$);
\node[draw=black, circle,  line width = .6pt, inner sep=1pt] at (A) {};
    \draw[thick] ($(A)+(.04,0)$) -- (B);  % Linea superiore del crescendo
    \draw[thick] ($(A)+(.04,0)$) -- (C); % Linea inferiore del crescendo
	\node at (D) {\Large \textit{f}};
\separatore{in6bis}{out6bis}{\cardHeight/4}

% ----------------------------	CARD	7
\separatore{in7}{out7}{\cardHeight/4}
\coordinate (A) at ($(card7) - (\cardwidth/4, \dinamicY)$);
\coordinate (B) at ($(A) + (\cardwidth/2.5, \cardHeight/25)$);
\coordinate (C) at ($(A) + (\cardwidth/2.5, -\cardHeight/25)$);
\coordinate (D) at ($(card7) - (-\cardwidth/4,\dinamicY)$);
\node at ($(A)-(0,.05)$) {\Large\textit{p}};
    \draw[thick] ($(A)+(.2,0)$) -- (B);  % Linea superiore del crescendo
    \draw[thick] ($(A)+(.2,0)$) -- (C); % Linea inferiore del crescendo
	\node at (D) {\Large \textit{f}};
% ----------------------------	CARD	8
\separatore{in8}{out8}{\cardHeight/4}
\coordinate (A) at ($(card8) - (0, \dinamicY)$);
\node at (A) {\Large\textit{mf}};
% ----------------------------	CARD	9
\separatore{in9}{out9}{\cardHeight/4}
\coordinate (A) at ($(card9) - (\cardwidth/3.5, \dinamicY)$);
\coordinate (B) at ($(A) + (\cardwidth/2.4, \cardHeight/25)$);
\coordinate (C) at ($(A) + (\cardwidth/2.4, -\cardHeight/25)$);
\coordinate (D) at ($(card9) - (-\cardwidth/3.5,\dinamicY)$);
\node at ($(A)-(0,.05)$) {\Large \textit{mp}};
    \draw[thick] ($(A)+(.4,0)$) -- (B);  % Linea superiore del crescendo
    \draw[thick] ($(A)+(.4,0)$) -- (C); % Linea inferiore del crescendo
	\node at (D) {\Large \textit{sfz}};
% ----------------------------	CARD	10
\separatore{in10}{out10}{\cardHeight/4}
    \node at ($(card10) - (0, \dinamicY)$) {\Large \textit{mp} -- \textit{f}};
% ----------------------------	CARD	11
\separatore{in11}{out11}{\cardHeight/4}
    \node at ($(card11) - (0, \dinamicY)$) {\Large \textit{p} -- \textit{ff}};
% ----------------------------	CARD	12
\coordinate (out12) at ($(in12) + (\cardwidth,0)$);
\separatore{in12}{out12}{\cardHeight/4}
    \node at ($(card12) - (0, \dinamicY)$) {\Large \textit{pp} -- \textit{p}};


%======================================================================










%======================================================================
% -----------------------------	definizione durate -------------------------------------------------------------
% ----------------------------	CARD	1
\separatore{in1}{out1}{-\cardHeight/3}
\node at (midPoint) (a) {60 -- 100 s};
% ----------------------------	CARD	2
% ----------------------------	CARD	3
\separatore{in3}{out3}{-\cardHeight/3}
\node at (midPoint) (a) {90 -- 140 s};
% ----------------------------	CARD	4
\separatore{in4}{out4}{-\cardHeight/3}
\node at (midPoint) (a) {eps -- 2 s};
% ----------------------------	CARD	5
%\separatore{in5}{out5}{-\cardHeight/3}
% ----------------------------	CARD	6
\separatore{in6}{out6}{-\cardHeight/3}
\node at (midPoint) (a) {1 -- 10 s};
% ----------------------------	CARD	6bis
\separatore{in6bis}{out6bis}{-\cardHeight/3}
\node at (midPoint) (a) {4 -- 15 s};
% ----------------------------	CARD	7
\separatore{in7}{out7}{-\cardHeight/3}
\node at (midPoint) (a) {3 -- 9 s};
% ----------------------------	CARD	8
\separatore{in8}{out8}{-\cardHeight/3}
\node at (midPoint) (a) {eps -- 4 s};
% ----------------------------	CARD	9
\separatore{in9}{out9}{-\cardHeight/3}
\node at (midPoint) (a) {4 -- 12 s};
% ----------------------------	CARD	10
\separatore{in10}{out10}{-\cardHeight/3}
\node at (midPoint) (a) {eps -- 4 s};
% ----------------------------	CARD	11
\separatore{in11}{out11}{-\cardHeight/3}
\node at (midPoint) (a) {eps -- 3 s};
% ----------------------------	CARD	12
\separatore{in12}{out12}{-\cardHeight/3}
\node at (midPoint) (a) {10 -- 30 s};
%======================================================================












%======================================================================
% -----------------------------	definizione ribattuti -------------------------------------------------------------
% ----------------------------	CARD	6
\coordinate (B) at  ($(card6) - (\cardwidth/14,\cardHeight/5)$);
\begin{scope}[rotate=20]
    \draw (B) -- ++(\cardwidth/7,0);
\end{scope}
\begin{scope}[rotate=20]
    \coordinate (C) at ($(B) + (\cardwidth/80,\cardHeight/80)$);
    \draw (C) -- ++(\cardwidth/7,0);
\end{scope}
\begin{scope}[rotate=20]
    \coordinate (C) at ($(B) + (\cardwidth/45,\cardHeight/40)$);
    \draw (C) -- ++(\cardwidth/7,0);
\end{scope}
% ----------------------------	CARD	6bis
\coordinate (B) at  ($(card6bis) - (\cardwidth/14,\cardHeight/5)$);
\begin{scope}[rotate=20]
    \draw (B) -- ++(\cardwidth/7,0);
\end{scope}
\begin{scope}[rotate=20]
    \coordinate (C) at ($(B) + (\cardwidth/80,\cardHeight/80)$);
    \draw (C) -- ++(\cardwidth/7,0);
\end{scope}
\begin{scope}[rotate=20]
    \coordinate (C) at ($(B) + (\cardwidth/45,\cardHeight/40)$);
    \draw (C) -- ++(\cardwidth/7,0);
\end{scope}
%======================================================================











%======================================================================
% -----------------------------	definizione FRECCE superiori --------------------------------------------------
%
% ----------------------------	CARD 4 => CARD 4
\def\offsetHeight{1.6};
\draw[arrows = {-Kite[open, width=\arrowWidth, length=\arrowLength, line width = \lineWidth]},line width = \lineWidth, rounded corners=10pt]   (a4.center) %to[out=77, in=100, looseness=3] (c3.center); 
-- ($(a4.center)+(0,\cardHeight-\offsetHeight)$) -- ($(c3.center)+(0,\cardHeight-\offsetHeight)$) -- (c3.center);
% ----------------------------	CARD 5 => CARD 4
\def\offsetHeight{1.2};
\coordinate (cardFiveArrowLeft) at ($(b3.center)+(0,\cardHeight - \offsetHeight)$);
\coordinate (cardFiveArrowRight) at ($(a5.center)+(0,\cardHeight-\offsetHeight)$);
\draw[arrows = {-Kite[open, width=\arrowWidth, length=\arrowLength, line width = \lineWidth]},line width = \lineWidth, rounded corners=10pt]   (a5.center) %to[out=80, in=100, looseness=2] (b3.center); 
-- (cardFiveArrowRight) -- (cardFiveArrowLeft) -- (b3.center);
% ----------------------------	CARD 6 => CARD 6
\def\offsetHeight{1.6};
\coordinate (cardSixArrowLeft) at ($(c5.center)+(0,\cardHeight-\offsetHeight)$);
\coordinate (cardSixArrowRight) at ($(a6.center)+(0,\cardHeight-\offsetHeight)$);
\draw[arrows = {-Kite[open, width=\arrowWidth, length=\arrowLength, line width = \lineWidth]},line width = \lineWidth, rounded corners=10pt]   (a6.center) %to[out=77, in=100, looseness=3] (c5.center); 
-- (cardSixArrowRight) -- (cardSixArrowLeft) -- (c5.center);
% ----------------------------	CARD 8 => CARD 7
\def\offsetHeight{1.6};
\coordinate (cardEightArrowLeft) at ($(c6.center)+(0,\cardHeight-\offsetHeight)$);
\coordinate (cardEightArrowRight) at ($(a8.center)+(0,\cardHeight-\offsetHeight)$);
\draw[arrows = {-Kite[open, width=\arrowWidth, length=\arrowLength, line width = \lineWidth]},line width = \lineWidth, rounded corners=10pt]   (a8.center) %to[out=80, in=100, looseness=2] (c6.center); 
-- (cardEightArrowRight) -- (cardEightArrowLeft) -- (c6.center);
% ----------------------------	CARD 11 => CARD 10
\def\offsetHeight{1.6};
\coordinate (cardTenArrowLeft) at ($(b9.center)+(0,\cardHeight-\offsetHeight)$);
\coordinate (cardTenArrowRight) at ($(a11.center)+(0,\cardHeight-\offsetHeight)$);
\draw[arrows = {-Kite[open, width=\arrowWidth, length=\arrowLength, line width = \lineWidth]},line width = \lineWidth, rounded corners=10pt]   (a11.center) -- (cardTenArrowRight) -- (cardTenArrowLeft) -- (b9.center);
%======================================================================

 
 
 
 
 
 
 
 
 
 %======================================================================
% -----------------------------	definizione FRECCE inferiori ----------------------------------------------------
% ----------------------------	CARD 8 => CARD 6bis
\coordinate (A) at ($(a5)-(0,\cardHeight*2.15)$);
    \gettikzxy{(b8)}{\bx}{\by}
    \gettikzxy{(out6bis)}{\outx}{\outy}
    \gettikzxy{(card7)}{\cardx}{\cardy}
    \gettikzxy{(out6bis)}{\outx}{\outy}
    \gettikzxy{(A)}{\ax}{\ay}
    \gettikzxy{(in6bis)}{\inx}{\iny}
\draw[arrows = {-Kite[open, width=\arrowWidth, length=\arrowLength, line width = \lineWidth]},line width = \lineWidth, rounded corners=10pt]  (b8.center) -- node[pos =0.4375,circle, draw, fill=black, inner sep=0mm, line width = \pointWidth] (a6bis) {} ( \bx,\outy) -- (\cardx, \outy)  -- (\cardx, \ay) -- (A) -- (\ax, \iny) -- (in6bis.center);
% ----------------------------	CARD 6bis => CARD 7
    \gettikzxy{(b6)}{\bx}{\by}
    \gettikzxy{(out6bis)}{\outx}{\outy}
\draw[arrows = {-Kite[open, width=\arrowWidth, length=\arrowLength, line width = \lineWidth]},line width = \lineWidth, rounded corners=10pt]   (out6bis.center) -- (\bx,\outy) -- (b6.center);
% ----------------------------	CARD 8 => CARD 6
    \gettikzxy{(c6)}{\cx}{\cy}
    \gettikzxy{(a6bis)}{\abisx}{\abisy}
    \gettikzxy{(a6)}{\ax}{\ay}
\coordinate (A) at (\cx,\abisy);
\coordinate (B) at (\ax,\abisy);
    \gettikzxy{(c5)}{\bx}{\by}
\coordinate (C) at (\bx,\abisy);    
\draw[line width = \lineWidth]  (a6bis)  -- (A);
\draw[line width = \lineWidth] ($(A) - (0,.0228cm)$) arc[start angle=-180, end angle=0, radius=-.3];
\draw[arrows = {-Kite[open, width=\arrowWidth, length=\arrowLength, line width = \lineWidth]},line width = \lineWidth, rounded corners=10pt]  (B) -- (C)-- (c5.center); 
% ----------------------------	CARD 9 => CARD 9
\def\offsetHeight{.8};
\draw[arrows = {-Kite[open, width=\arrowWidth, length=\arrowLength, line width = \lineWidth]},line width = \lineWidth, rounded corners=10pt]   (a9.center) -- ($(a9)-(0,\cardHeight/2+\offsetHeight)$) -- ($(c8)-(0,\cardHeight/2+\offsetHeight)$) -- (c8.center);
% ----------------------------	CARD 11 => CARD 10
\draw[arrows = {-Kite[open, width=\arrowWidth, length=\arrowLength, line width = \lineWidth]},line width = \lineWidth, rounded corners=10pt]   (a11.center) %to[out=-80, in=-100, looseness=1.5] (b9.center);
 -- ($(a11)-(0,\cardHeight/2+\offsetHeight)$) -- ($(b9)-(0,\cardHeight/2+\offsetHeight)$) -- (b9.center);
% ----------------------------	CARD 11 => CARD 3
    \gettikzxy{(b11)}{\bx}{\by}
    \gettikzxy{(out7)}{\outx}{\outy}
\coordinate (A) at (\outx,\by);
\coordinate (B) at ($(A)-(0,\cardHeight/2)$);
    \gettikzxy{(B)}{\Bx}{\By}
    \gettikzxy{(a3)}{\ax}{\ay}
\coordinate (C) at (\ax,\By);
\draw[arrows = {-Kite[open, width=\arrowWidth, length=\arrowLength, line width = \lineWidth]},line width = \lineWidth, rounded corners=10pt]   (b11.center)  -- (A) -- (B) -- (C) -- (a3.center); 
%======================================================================








%======================================================================
% -----------------------------	riempio i nodi ----------------------------------------------------

    \fill[black] (a4) circle[radius=\pointWidth];
    \fill[black] (a5) circle[radius=\pointWidth];
    \fill[black] (a6) circle[radius=\pointWidth];
    \fill[black] (a8) circle[radius=\pointWidth];
    \fill[black] (b8) circle[radius=\pointWidth];
    \fill[black] (a9) circle[radius=\pointWidth];
    \fill[black] (a11) circle[radius=\pointWidth];
    \fill[black] (b11) circle[radius=\pointWidth];
    \fill[black] (a6bis) circle[radius=\pointWidth];
%======================================================================








%======================================================================
% -----------------------------	sezioni --------------------------------------------------------------------------------

% ----------------------------	CARD 4 ~ CARD 5	SEZIONE A 
\coordinate (upL) at ($(cardFiveArrowLeft)+(-.2cm,.6cm)$);
\coordinate (upR) at ($(cardFiveArrowRight)+(.2cm,.6cm)$);
\node[right,] at (upL) {\Large [ A ]};
\coordinate (newUpL) at ($(upL) + (1.2cm, 0)$); 
    \gettikzxy{(upL)}{\upLx}{\upLy}
    \gettikzxy{(upR)}{\upRx}{\upRy}
\coordinate (downL) at ($(\upLx,\paperHeight)-(0,\upLy)$);
\coordinate (downR) at ($(\upRx,\paperHeight)-(0,\upRy)$);
\draw[dashed, line width = 1pt, rounded corners=10pt] (newUpL) -- (upR) -- (downR) -- (downL) -- (upL);
% ----------------------------	CARD 5 ~ CARD 8	SEZIONE B
\coordinate (upL) at ($(cardSixArrowLeft)+(-.2cm,1cm)$);
\coordinate (upR) at ($(cardEightArrowRight)+(.5cm,1cm)$);
\node[right,] at (upL) {\Large [ B ]};
\coordinate (newUpL) at ($(upL) + (1.2cm, 0)$); 
    \gettikzxy{(upL)}{\upLx}{\upLy}
    \gettikzxy{(upR)}{\upRx}{\upRy}
\coordinate (downL) at ($(\upLx,\paperHeight)-(0,\upLy)$);
\coordinate (downR) at ($(\upRx,\paperHeight)-(0,\upRy)$);
\draw[dashed, line width = 1pt, rounded corners=10pt] (newUpL) -- (upR) -- (downR) -- (downL) -- (upL);
% ----------------------------	CARD 10 ~ CARD 11	SEZIONE C
\coordinate (upL) at ($(cardTenArrowLeft)+(-.3cm,1cm)$);
\coordinate (upR) at ($(cardTenArrowRight)+(.4cm,1cm)$);
\node[right,] at (upL) {\Large [ C ]};
\coordinate (newUpL) at ($(upL) + (1.2cm, 0)$); 
    \gettikzxy{(upL)}{\upLx}{\upLy}
    \gettikzxy{(upR)}{\upRx}{\upRy}
\coordinate (downL) at ($(\upLx,\paperHeight)-(0,\upLy)$);
\coordinate (downR) at ($(\upRx,\paperHeight)-(0,\upRy)$);
\draw[dashed, line width = 1pt, rounded corners=10pt] (newUpL) -- (upR) -- (downR) -- (downL) -- (upL);

%======================================================================











\end{tikzpicture}
\end{document}
