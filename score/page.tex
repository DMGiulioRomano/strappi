\documentclass[tikz, a3paper, landscape, border=0]{standalone}
\usetikzlibrary{calc}
\usepackage{comandi}
\usepackage{musicography}
\usepackage{pgfmath}
\begin{document}
\begin{tikzpicture}

%======================================================================
% ------------------------------ funzione per ottenere x e y di un nodo ----------------------------------------
\makeatletter
\newcommand{\gettikzxy}[3]{%
  \tikz@scan@one@point\pgfutil@firstofone#1\relax
  \edef#2{\the\pgf@x}%
  \edef#3{\the\pgf@y}%
}
\makeatother
%======================================================================







%======================================================================
% ------------------------------ preparazione foglio di lavoro ----------------------------------------------------
\def\paperHeight{29.7}
\def\paperWidth{42}
\fill[white] (0,0) rectangle ++ (\paperWidth,\paperHeight);
% Imposta la griglia del foglio A3 come riferimento (42x29.7 cm)
%\draw[step=.1cm, gray, very thin] (0, 0) grid (\paperWidth, \paperHeight); % Griglia per visualizzare lo spazio A3
%\draw[step=1cm, black, very thin] (0, 0) grid (\paperWidth, \paperHeight); % Griglia per visualizzare lo spazio A3
%======================================================================











%======================================================================
% ------------------------------ definizione variabili per sezione centrale del brano -----------------------
\def\cardHeight{5} 
\def\cardwidth{3} 
\def\cardSpacing{1.7}
\def\cardOffset{-2.3}
\def\cardY{\paperHeight/2-\cardHeight/2}
% ------------------
\def\battenteheight{.25}
\def\battentewidth{1.3}
% ------------------
\def\pointWidth{3pt}
\def\lineWidth{1.3pt}
\def\minimumSize{1.5pt}
% ------------------
\def\arrowWidth{7pt}
\def\arrowLength{10pt}
\def\sumWidth{1pt}
\def\roundedCorners{8pt}
%======================================================================












%======================================================================
% -----------------------------	carte		------------------------------------------------
%
\foreach \i  in {1,...,3}{
        		\edef\cardname{card\i}
        		\edef\cardin{in\i}
        		\edef\cardout{out\i}
        		\card{-4+\cardSpacing*\cardwidth*\i}{\cardY+10}{\cardwidth}{\cardHeight}{\cardname}{\cardin}{\cardout}
}
% ----------------------------	CARD	4
\card{\cardOffset+\cardSpacing*\cardwidth}{\cardY}{\cardwidth}{\cardHeight}{card4}{in4}{out4}
% ----------------------------	CARD	5
\card{\cardOffset+\cardSpacing*\cardwidth*1.8}{\cardY}{\cardwidth}{\cardHeight}{card5}{in5}{out5}
% ----------------------------	CARD	6
\card{\cardOffset+\cardSpacing*\cardwidth*2.95}{\cardY}{\cardwidth}{\cardHeight}{card6}{in6}{out6}
% ----------------------------	CARD	6bis
\card{\cardOffset+\cardSpacing*\cardwidth*2.95}{\cardY-\cardHeight*1.2}{\cardwidth}{\cardHeight}{card6bis}{in6bis}{out6bis}
% ----------------------------	CARD	7
\card{\cardOffset+\cardSpacing*\cardwidth*3.95}{\cardY}{\cardwidth}{\cardHeight}{card7}{in7}{out7}
% ----------------------------	CARD	8
\card{\cardOffset+\cardSpacing*\cardwidth*4.65}{\cardY}{\cardwidth}{\cardHeight}{card8}{in8}{out8}
% ----------------------------	CARD	9
\card{\cardOffset+\cardSpacing*\cardwidth*5.8}{\cardY}{\cardwidth}{\cardHeight}{card9}{in9}{out9}
% ----------------------------	CARD	10
\card{\cardOffset+\cardSpacing*\cardwidth*6.8}{\cardY}{\cardwidth}{\cardHeight}{card10}{in10}{out10}
% ----------------------------	CARD	11
\card{\cardOffset+\cardSpacing*\cardwidth*7.5}{\cardY}{\cardwidth}{\cardHeight}{card11}{in11}{out11}
% ----------------------------	CARD	12
\card{23+\cardSpacing*\cardwidth*3}{\cardY-10}{\cardwidth}{\cardHeight}{card12}{in12}{noout}
%====================================================================

















%======================================================================
% -----------------------------	definizione campane		-------------------------------------------------------
%
\campana{0}{0}{.8}{capovolta}{libera}{card1}
\campana{0}{0}{.8}{capovolta}{libera}{card2}
\campana{0}{0}{.8}{ordinaria}{bloccata}{card3}
\campana{0}{0}{.8}{ordinaria}{bloccata}{card4}
\campana{0}{0}{.8}{ordinaria}{libera}{card5}
\campana{0}{0}{.8}{ordinaria}{libera}{card6}
\campanaAcc{0}{0}{.8}{bloccata}{card6bis}
\campanaAcc{0}{0}{.8}{libera}{card7}
\campana{0}{0}{.8}{ordinaria}{libera}{card8}
\campana{0}{0}{.8}{ordinaria}{libera}{card9}
\campanaAcc{0}{0}{.8}{libera}{card10}
\campanaAcc{0}{0}{.8}{libera}{card11}
\campana{0}{0}{.8}{ordinaria}{bloccata}{card12}
%======================================================================









%======================================================================
% -----------------------------	definizione battenti		-------------------------------------------------------
%
% ----------------------------	CARD	1
\updateBattentePosition{0.1}{.3}
\battente{card1}{\battenteStartX}{\battenteStartY}{\battentewidth}{\battenteheight}{libero}{60}
\strofinato{card1}{.65}{.1}{-1.15}{.7}
% ----------------------------	CARD	2
\updateBattentePosition{0}{.53}
\battente{card2}{\battenteStartX}{\battenteStartY}{\battentewidth}{\battenteheight}{libero}{0}
\draw[arrows = {-Latex}, double distance=1.2pt] ($(card2) + (0,1.6)$) -- ($(card2) + (0,.63)$);
% ----------------------------	CARD	3
\updateBattentePosition{0}{.32}
\battente{card3}{\battenteStartX}{\battenteStartY}{\battentewidth}{\battenteheight}{libero}{60}
\strofinato{card3}{.8}{.1}{-1.45}{.7}
% ----------------------------	CARD	4
\updateBattentePosition{-.35}{.32}
\battente{card4}{\battenteStartX}{\battenteStartY}{\battentewidth}{\battenteheight}{libero}{45}
% ----------------------------	CARD	5
\updateBattenteAccPosition{0}{.53}
\battenteAcc{card5}{\battenteStartAccX}{\battenteStartAccY}{\battentewidth}{\battenteheight}{aperto}{0}
\draw[arrows = {-Latex}, double distance=1.2pt] ($(card5) + (0,1.6)$) -- ($(card5) + (0,.63)$);
% ----------------------------	CARD	6
\updateBattentePosition{-.45}{.0}
\battente{card6}{\battenteStartX}{\battenteStartY}{\battentewidth}{\battenteheight}{bloccato}{0}
% ----------------------------	CARD	7
\updateBattentePosition{0}{.32}
\battente{card7}{\battenteStartX}{\battenteStartY}{\battentewidth}{\battenteheight}{bloccato}{60}
\strofinato{card7}{.8}{.1}{-1.45}{.7}
% ----------------------------	CARD	8
\updateBattentePosition{-.35}{.32}
\battente{card8}{\battenteStartX}{\battenteStartY}{\battentewidth}{\battenteheight}{bloccato}{45}
% ----------------------------	CARD	9
\updateBattentePosition{0}{.32}
\battente{card9}{\battenteStartX}{\battenteStartY}{\battentewidth}{\battenteheight}{libero}{60}
\strofinato{card9}{.8}{.1}{-1.45}{.7}
% ----------------------------	CARD	10
\updateBattentePosition{-.45}{.0}
\battente{card10}{\battenteStartX}{\battenteStartY}{\battentewidth}{\battenteheight}{libero}{0}
% ----------------------------	CARD	11
\updateBattentePosition{-.35}{.32}
\battente{card11}{\battenteStartX}{\battenteStartY}{\battentewidth}{\battenteheight}{bloccato}{45}
% ----------------------------	CARD	12
\updateBattentePosition{0.36}{-.1}
\battente{card12}{\battenteStartX}{\battenteStartY}{\battentewidth}{\battenteheight}{libero}{80}
\draw[line width = .8pt] ($(card12)+(-.8,+.4)$) -- ($(card12)+(+.8,+.4)$);
\draw[->]   ($(card12)+(-.5,-.4)$) to[out=-30, in=200, looseness=1] ($(card12)+(.6,-.4)$);
\campanaSbiadita{0}{0}{.8}{ordinaria}{bloccata}{card12}
%======================================================================




% ----------------------------	SEPARATORE
\coordinate (out12) at ($(in12) + (\cardwidth,0)$);
\newcommand{\separatore}[3]{
\draw[draw=black,  thin] ($(#1) - (0, #3)$) -- ($(#2) - (0,#3)$);
\coordinate (midPoint) at ($(#1) - (-\cardwidth/2, #3 + #3/4)$);
}


%======================================================================
% -----------------------------	definizione durate -------------------------------------------------------------
% ----------------------------	CARD	1
\separatore{in1}{out1}{-\cardHeight/3}
\node at (midPoint) (a) {\normalsize 30 s $\sim$};
% ----------------------------	CARD	2
\separatore{in2}{out2}{-\cardHeight/3}
\node at (midPoint) (a) {\normalsize epsylon};
% ----------------------------	CARD	3
\separatore{in3}{out3}{-\cardHeight/3}
\node at (midPoint) (a) {\normalsize 30 s $\sim$};
% ----------------------------	CARD	4
\separatore{in4}{out4}{-\cardHeight/3}
\node at (midPoint) (a) {\normalsize eps $\sim$ 2 s};
% ----------------------------	CARD	5
\separatore{in5}{out5}{-\cardHeight/3}
\node at (midPoint) (a) {\normalsize epsylon};
% ----------------------------	CARD	6
\separatore{in6}{out6}{-\cardHeight/3}
\node at (midPoint) (a) {\normalsize 1 $\sim$ 10 s};
% ----------------------------	CARD	6bis
\separatore{in6bis}{out6bis}{-\cardHeight/3}
\node at (midPoint) (a) {\normalsize 4 $\sim$ 15 s};
% ----------------------------	CARD	7
\separatore{in7}{out7}{-\cardHeight/3}
\node at (midPoint) (a) {\normalsize 3 $\sim$ 9 s};
% ----------------------------	CARD	8
\separatore{in8}{out8}{-\cardHeight/3}
\node at (midPoint) (a) {\normalsize eps $\sim$ 4 s};
% ----------------------------	CARD	9
\separatore{in9}{out9}{-\cardHeight/3}
\node at (midPoint) (a) {\normalsize 4 $\sim$ 12 s};
% ----------------------------	CARD	10
\separatore{in10}{out10}{-\cardHeight/3}
\node at (midPoint) (a) {\normalsize eps $\sim$ 4 s};
% ----------------------------	CARD	11
\separatore{in11}{out11}{-\cardHeight/3}
\node at (midPoint) (a) {\normalsize eps $\sim$ 3 s};
% ----------------------------	CARD	12
\separatore{in12}{out12}{-\cardHeight/3}
\node at (midPoint) (a) {\normalsize 10 $\sim$ 30 s};
%======================================================================






%======================================================================
% -----------------------------	definizione dinamiche -------------------------------------------------------------
\def\dinamicY{\cardHeight/2.45}
% ----------------------------	CARD	1
\separatore{in1}{out1}{\cardHeight/3}
    \node at ($(card1) - (0, \dinamicY)$) {\normalsize \textit{mf} $\sim$ \textit{ff}};
% ----------------------------	CARD	2
\separatore{in2}{out2}{\cardHeight/3}
    \node at ($(card2) - (0, \dinamicY)$) {\normalsize \textit{mf} };
% ----------------------------	CARD	3
\separatore{in3}{out3}{\cardHeight/3}
    \node at ($(card3) - (0, \dinamicY)$) {\normalsize \textit{mf} $\sim$ \textit{f}};
% ----------------------------	CARD	4
\separatore{in4}{out4}{\cardHeight/3}
    \node at ($(card4) - (0, \dinamicY)$) {\normalsize \textit{mp} $\sim$ \textit{mf}};
% ----------------------------	CARD	5
\separatore{in5}{out5}{\cardHeight/3}
    \node at ($(card5) - (0, \dinamicY)$) {\normalsize \textit{pp} $\sim$ \textit{mf}};
% ----------------------------	CARD	6
\separatore{in6}{out6}{\cardHeight/3}
\coordinate (A) at ($(card6) - (\cardwidth/4, \dinamicY)$);
\coordinate (B) at ($(A) + (\cardwidth/2.5, \cardHeight/35)$);
\coordinate (C) at ($(A) + (\cardwidth/2.5, -\cardHeight/35)$);
\coordinate (D) at ($(card6) - (-\cardwidth/4,\dinamicY)$);
\node[draw=black, circle,  thin, inner sep=1pt] at (A) {};
    \draw[thin] ($(A)+(.05,0)$) -- (B);  % Linea superiore del crescendo
    \draw[thin] ($(A)+(.05,0)$) -- (C); % Linea inferiore del crescendo
	\node at (D) {\normalsize \textit{f}};
% ----------------------------	CARD	6bis
\separatore{in6bis}{out6bis}{\cardHeight/3}
\coordinate (A) at ($(card6bis) - (\cardwidth/4, \dinamicY)$);
\coordinate (B) at ($(A) + (\cardwidth/2.5, \cardHeight/35)$);
\coordinate (C) at ($(A) + (\cardwidth/2.5, -\cardHeight/35)$);
\coordinate (D) at ($(card6bis) - (-\cardwidth/4,\dinamicY)$);
\node[draw=black, circle,  thin, inner sep=1pt] at (A) {};
    \draw[thin] ($(A)+(.05,0)$) -- (B);  % Linea superiore del crescendo
    \draw[thin] ($(A)+(.05,0)$) -- (C); % Linea inferiore del crescendo
	\node at (D) {\normalsize \textit{f}};

% ----------------------------	CARD	7
\separatore{in7}{out7}{\cardHeight/3}
\coordinate (A) at ($(card7) - (\cardwidth/4, \dinamicY)$);
\coordinate (B) at ($(A) + (\cardwidth/2.5, \cardHeight/35)$);
\coordinate (C) at ($(A) + (\cardwidth/2.5, -\cardHeight/35)$);
\coordinate (D) at ($(card7) - (-\cardwidth/4,\dinamicY)$);
\node at ($(A)$) {\normalsize\textit{mf}};
    \draw[thin] ($(A)+(.3,0)$) -- (B);  % Linea superiore del crescendo
    \draw[thin] ($(A)+(.3,0)$) -- (C); % Linea inferiore del crescendo
	\node at (D) {\normalsize \textit{ff}};
% ----------------------------	CARD	8
\separatore{in8}{out8}{\cardHeight/3}
\coordinate (A) at ($(card8) - (0, \dinamicY)$);
\node at (A) {\normalsize \textit{mf}};
% ----------------------------	CARD	9
\separatore{in9}{out9}{\cardHeight/3}
\coordinate (A) at ($(card9) - (\cardwidth/3.5, \dinamicY)$);
\coordinate (B) at ($(A) + (\cardwidth/2.4, \cardHeight/35)$);
\coordinate (C) at ($(A) + (\cardwidth/2.4, -\cardHeight/35)$);
\coordinate (D) at ($(card9) - (-\cardwidth/3.5,\dinamicY)$);
\node at ($(A)$) {\normalsize \textit{mf}};
    \draw[thin] ($(A)+(.4,0)$) -- (B);  % Linea superiore del crescendo
    \draw[thin] ($(A)+(.4,0)$) -- (C); % Linea inferiore del crescendo
	\node at (D) {\normalsize \textit{sfz}};
% ----------------------------	CARD	10
\separatore{in10}{out10}{\cardHeight/3}
    \node at ($(card10) - (0, \dinamicY)$) {\normalsize \textit{mp} $\sim$ \textit{f}};
% ----------------------------	CARD	11
\separatore{in11}{out11}{\cardHeight/3}
    \node at ($(card11) - (0, \dinamicY)$) {\normalsize \textit{p} $\sim$ \textit{ff}};
% ----------------------------	CARD	12
\coordinate (out12) at ($(in12) + (\cardwidth,0)$);
\separatore{in12}{out12}{\cardHeight/3}
    \node at ($(card12) - (0, \dinamicY)$) {\normalsize \textit{pp} $\sim$ \textit{p}};


%======================================================================





















%======================================================================
% -----------------------------	definizione FRECCE e NODI carte --------------------------------------------
%	
% ----------------------------	CARD	1 - 2
	\foreach \i [evaluate=\i as \next using int(\i+1)] in {1,...,2}{
        		\edef\cardin{in\next}
        		\edef\cardout{out\i}
		\edef\a{a\i}
		\edef\b{b\i}
		\edef\c{c\i}	
		\draw[arrows = {-Stealth[black, width=\arrowWidth, length=\arrowLength, line width = \lineWidth]},line width = \lineWidth]  (\cardout.center) -- (\cardin.center);
	}
% ----------------------------	CARD	3
    \coordinate (Z) at ($(out3.center) + (1,0)$);
    \coordinate (A) at ($(Z) - (0,\cardHeight/1.2)$);
    \coordinate (B) at ($(A) - (\cardwidth*4.9,0)$);
    \gettikzxy{(B)}{\bx}{\by}
    \gettikzxy{(in4)}{\inx}{\iny}
    \coordinate (C) at (\bx,\iny);
    \draw[arrows = {-Stealth[black, width=\arrowWidth, length=\arrowLength, line width = \lineWidth]},line width = \lineWidth, rounded corners=\roundedCorners] 	(out3.center) -- (Z) -- (A)-- (B) -- (C) -- node[pos=.3, name=a3, circle, draw, fill=white, minimum size=1pt, inner sep = .05cm, line width=\sumWidth] {\scriptsize +}  node[circle, draw, fill=white, minimum size=1pt, pos=0.65, name=b3, inner sep = .05cm, line width=\sumWidth] {\scriptsize +} (in4.center);
% ----------------------------	CARD	5
\def\i{5}
\def\nexti{6}
\draw[arrows = {-Stealth[black, width=\arrowWidth, length=\arrowLength, line width = \lineWidth]},line width = \lineWidth, rounded corners=\roundedCorners]  (out\i.center) -- (in\nexti.center)
	node[pos =0.15,circle, draw, fill=black, inner sep=0mm, line width = \pointWidth, minimum size=\minimumSize] (a\i) {}
        	node[pos=.55, inner sep=0mm, line width = \pointWidth, circle, draw, fill=white, minimum size=1pt, inner sep = .05cm, line width=\sumWidth] (b\i) {\scriptsize +}
        	node[pos=.75, inner sep=0mm, line width = \pointWidth, circle, draw, fill=white, minimum size=1pt, inner sep = .05cm, line width=\sumWidth] (c\i) {\scriptsize +};
% ----------------------------	CARD	4
\def\i{4}
\def\nexti{5}
\gettikzxy{(b3)}{\bx}{\by}
\gettikzxy{(a5)}{\ax}{\ay}
\draw[arrows = {-Stealth[black, width=\arrowWidth, length=\arrowLength, line width = \lineWidth]},line width = \lineWidth, rounded corners=\roundedCorners]  (out\i.center) --  (in\nexti.center)
	node[pos =0.5,circle, draw, fill=black, inner sep=0mm, minimum size=\minimumSize, line width = \pointWidth] (a\i) {};
% ----------------------------	CARD	6
\def\i{6}
\def\nexti{7}
\draw[arrows = {-Stealth[black, width=\arrowWidth, length=\arrowLength, line width = \lineWidth]},line width = \lineWidth]  (out\i.center) -- (in\nexti.center)
	node[pos =0.2,circle, draw, fill=black, inner sep=0mm, line width = \pointWidth, minimum size = \minimumSize] (a\i) {}
        	node[pos=.42, circle, draw, fill=white, minimum size=1pt, inner sep = .05cm, line width=\sumWidth] (b\i) {\scriptsize +} 
        	node[pos=.7, circle, draw, fill=white, minimum size=1pt, inner sep = .05cm, line width=\sumWidth] (c\i) {\scriptsize +};
% ----------------------------	CARD	7
\def\i{7}
\def\nexti{8}
\draw[arrows = {-Stealth[black, width=\arrowWidth, length=\arrowLength, line width = \lineWidth]},line width = \lineWidth]  (out\i.center) -- (in\nexti.center);
% ----------------------------	CARD	8
\def\i{8}
\def\nexti{9}
\draw[arrows = {-Stealth[black, width=\arrowWidth, length=\arrowLength, line width = \lineWidth]},line width = \lineWidth]  (out\i.center) -- (in\nexti.center)
	node[pos =0.2,circle, draw, fill=black, inner sep=0mm, line width = \pointWidth,minimum size = \minimumSize] (a\i) {}
        	node[pos=.35,circle, draw, fill=black, inner sep=0mm, line width = \pointWidth,minimum size = \minimumSize] (b\i) {}
        	node[pos=.75,circle, draw, fill=white, minimum size=1pt, inner sep = .05cm, line width=\sumWidth] (c\i) {\scriptsize +};
% ----------------------------	CARD	9
\def\i{9}
\def\nexti{10}
\draw[arrows = {-Stealth[black, width=\arrowWidth, length=\arrowLength, line width = \lineWidth]},line width = \lineWidth]  (out\i.center) -- (in\nexti.center)
	node[pos =0.2,circle, draw, fill=black, inner sep=0mm, line width = \pointWidth, minimum size = \minimumSize] (a\i) {}
        	node[fill=white,circle, draw, pos=.42,inner sep=.05cm, minimum size=1pt, line width = \pointWidth, line width=\sumWidth] (b\i) {\scriptsize +}        	
	node[fill=white,circle, draw, pos=.7, inner sep=.05cm, minimum size=1pt, line width = \pointWidth, line width=\sumWidth] (c\i) {\scriptsize +};
% ----------------------------	CARD	10
\def\i{10}
\def\nexti{11}
\draw[arrows = {-Stealth[black, width=\arrowWidth, length=\arrowLength, line width = \lineWidth]},line width = \lineWidth]  (out\i.center) -- (in\nexti.center);
% ----------------------------	CARD	11
\gettikzxy{(out11)}{\outx}{\outy}
\gettikzxy{(in11)}{\inx}{\iny}
\gettikzxy{(card12)}{\cardx}{\cardy}
\gettikzxy{(card10)}{\cardtenx}{\cardteny}

\coordinate (A) at (\cardx + 1cm,\outy);
\coordinate (B) at ($(A) - (0,\cardHeight*1.2)$);
\gettikzxy{(B)}{\bx}{\by}
\coordinate (C) at (\cardtenx,\by);
\coordinate (D) at (\cardtenx,\cardy);

\draw[arrows = {-Stealth[black, width=\arrowWidth, length=\arrowLength, line width = \lineWidth]},line width = \lineWidth,  rounded corners=\roundedCorners, minimum size = \minimumSize] 	(out11.center) -- node[pos =0.3,circle, draw, fill=black, inner sep=0mm, line width = \pointWidth] (a11) {} (A) -- (B) -- (C) -- node[pos =0.8,circle, draw, fill=black, inner sep=0mm, line width = \pointWidth] (b11) {} (D) -- (in12.center);
%======================================================================
































%======================================================================
% -----------------------------	definizione FRECCE superiori --------------------------------------------------
%
% ----------------------------	CARD 5 => CARD 4
\def\offsetHeight{1.6};
\coordinate (cardFiveArrowLeft) at ($(b3.center)+(0,\cardHeight - \offsetHeight)$);
\coordinate (cardFiveArrowRight) at ($(a5.center)+(0,\cardHeight-\offsetHeight)$);
\gettikzxy{(cardFiveArrowRight)}{\cardFiveArrowRightx}{\cardFiveArrowRighty}
\gettikzxy{(a4)}{\afourx}{\afoury}
\node[circle, draw, fill=white, minimum size=1pt, name=(forArrow), inner sep = .05cm,line width =\sumWidth] at (\afourx,\cardFiveArrowRighty) (forArrow) {\scriptsize +};
%start arrow
\draw[arrows = {-Stealth[width=\arrowWidth, length=\arrowLength, line width = \lineWidth, ]},line width = \lineWidth, rounded corners=\roundedCorners]   (a5.center)  -- (cardFiveArrowRight) -- (forArrow);
%end arrow
\draw[arrows = {-Stealth[ width=\arrowWidth, length=\arrowLength, line width = \lineWidth]},line width = \lineWidth, rounded corners=\roundedCorners] (forArrow)
--  (cardFiveArrowLeft) -- (b3);
% ----------------------------	CARD 4 => freccia (CARD 4 => CARD5)
\def\offsetHeight{1.6};
\gettikzxy{(forArrow)}{\forArrowx}{\forArrowy}
\gettikzxy{(card4)}{\bx}{\cardy}
\draw[arrows = {-Stealth[ width=\arrowWidth, length=\arrowLength, line width = \lineWidth]},line width = \lineWidth, rounded corners=\roundedCorners]   (a4.center) -- (forArrow);
% ----------------------------	CARD 6 => CARD 6
\def\offsetHeight{1.6};
\coordinate (cardSixArrowLeft) at ($(c5.center)+(0,\cardHeight-\offsetHeight)$);
\coordinate (cardSixArrowRight) at ($(a6.center)+(0,\cardHeight-\offsetHeight)$);
\draw[arrows = {-Stealth[ width=\arrowWidth, length=\arrowLength, line width = \lineWidth]},line width = \lineWidth, rounded corners=\roundedCorners]   (a6.center)  
-- (cardSixArrowRight) -- (cardSixArrowLeft) -- (c5);
% ----------------------------	CARD 8 => CARD 7
\def\offsetHeight{1.6};
\coordinate (cardEightArrowLeft) at ($(c6.center)+(0,\cardHeight-\offsetHeight)$);
\coordinate (cardEightArrowRight) at ($(a8.center)+(0,\cardHeight-\offsetHeight)$);
\draw[arrows = {-Stealth[ width=\arrowWidth, length=\arrowLength, line width = \lineWidth]},line width = \lineWidth, rounded corners=\roundedCorners]   (a8.center)  
-- (cardEightArrowRight) -- (cardEightArrowLeft) -- (c6);
% ----------------------------	CARD 11 => CARD 10
\def\offsetHeight{1.6};
\coordinate (cardTenArrowLeft) at ($(c9.center)+(0,\cardHeight-\offsetHeight)$);
\coordinate (cardTenArrowRight) at ($(a11.center)+(0,\cardHeight-\offsetHeight)$);
\draw[arrows = {-Stealth[ width=\arrowWidth, length=\arrowLength, line width = \lineWidth]},line width = \lineWidth, rounded corners=\roundedCorners]   (a11.center) -- (cardTenArrowRight) -- (cardTenArrowLeft) -- (c9);
%======================================================================










 %======================================================================
% -----------------------------	definizione FRECCE inferiori ----------------------------------------------------
% ----------------------------	CARD 8 => CARD 6bis
\coordinate (A) at ($(c5)-(0,\cardHeight*1.8)$);
    \gettikzxy{(b8)}{\bx}{\by}
    \gettikzxy{(out6bis)}{\outx}{\outy}
    \gettikzxy{(card7)}{\cardx}{\cardy}
    \gettikzxy{(out6bis)}{\outx}{\outy}
    \gettikzxy{(A)}{\ax}{\ay}
    \gettikzxy{(in6bis)}{\inx}{\iny}
\draw[arrows = {-Stealth[ width=\arrowWidth, length=\arrowLength, line width = \lineWidth]},line width = \lineWidth, rounded corners=\roundedCorners]  (b8.center) -- node[pos =0.5,circle, draw, fill=black, inner sep=0mm, line width = \pointWidth, minimum size=\minimumSize] (a6bis) {} ( \bx,\outy) -- (\bx,\ay) -- (A) -- (\ax, \iny) -- (in6bis.center);
% ----------------------------	CARD 6bis => CARD 7
    \gettikzxy{(b6)}{\bx}{\by}
    \gettikzxy{(out6bis)}{\outx}{\outy}
\draw[arrows = {-Stealth[ width=\arrowWidth, length=\arrowLength, line width = \lineWidth]},line width = \lineWidth, rounded corners=\roundedCorners]   (out6bis.center) -- (\bx,\outy) -- (b6);
% ----------------------------	CARD 8 => CARD 6
    \gettikzxy{(c6)}{\cx}{\cy}
    \gettikzxy{(a6bis)}{\abisx}{\abisy}
    \gettikzxy{(a6)}{\ax}{\ay}
\coordinate (A) at (\cx-.28cm,\abisy);
\coordinate (B) at (\ax+.19cm,\abisy);
    \gettikzxy{(b5)}{\bx}{\by}
\coordinate (C) at (\bx,\abisy);    
\draw[line width = \lineWidth]  (a6bis)  -- (A);
\draw[line width = \lineWidth] ($(A) - (0,.0228cm)$) arc[start angle=-180, end angle=0, radius=-.3];
\draw[arrows = {-Stealth[ width=\arrowWidth, length=\arrowLength, line width = \lineWidth]},line width = \lineWidth, rounded corners=\roundedCorners]  (B) -- (C)-- (b5); 
% ----------------------------	CARD 9 => CARD 9
\def\offsetHeight{.8};
\draw[arrows = {-Stealth[ width=\arrowWidth, length=\arrowLength, line width = \lineWidth]},line width = \lineWidth, rounded corners=10pt]   (a9.center) -- ($(a9)-(0,\cardHeight/2+\offsetHeight)$) -- ($(c8)-(0,\cardHeight/2+\offsetHeight)$) -- (c8);
% ----------------------------	CARD 11 => CARD 10
\draw[arrows = {-Stealth[ width=\arrowWidth, length=\arrowLength, line width = \lineWidth]},line width = \lineWidth, rounded corners=10pt]   (a11) -- ($(a11)-(0,\cardHeight/2+\offsetHeight)$) -- ($(b9)-(0,\cardHeight/2+\offsetHeight)$) -- (b9);
% ----------------------------	CARD 11 => CARD 3
    \gettikzxy{(b11)}{\bx}{\by}
    \gettikzxy{(b9)}{\outx}{\outy}
\coordinate (A) at (\outx,\by);
\coordinate (B) at ($(A)-(0,\cardHeight/2)$);
    \gettikzxy{(B)}{\Bx}{\By}
    \gettikzxy{(a3)}{\ax}{\ay}
\coordinate (C) at (\ax,\By);
\draw[arrows = {-Stealth[width=\arrowWidth, length=\arrowLength, line width = \lineWidth]},line width = \lineWidth, rounded corners=10pt]   (b11.center)  -- (A) -- (B) -- (C) -- (a3); 
%======================================================================













%======================================================================
% -----------------------------	sezioni --------------------------------------------------------------------------------
\def\sezYup{\paperHeight*.5 + \cardHeight}
\def\sezYdown{\paperHeight*.5 - \cardHeight}
% ----------------------------	CARD 4 ~ CARD 5	SEZIONE A 
\coordinate (upL) at ($(cardFiveArrowLeft)+(-.4cm,.6cm)$);
\coordinate (upR) at ($(cardFiveArrowRight)+(.35cm,.6cm)$);
    \gettikzxy{(upL)}{\upLx}{\upLy}
    \gettikzxy{(upR)}{\upRx}{\upRy}
\node[above right] at (upL) {\Large [ A ]};
\coordinate (downL) at (\upLx,\sezYdown);
\coordinate (downR) at (\upRx,\sezYdown);
\coordinate (newUpL) at (\upLx,\sezYup);
\coordinate (newUpR) at (\upRx,\sezYup);
\draw[dashed, line width = 1pt, rounded corners=10pt] (newUpL) -- (newUpR) -- (downR) -- (downL) -- (upL) -- cycle;
% ----------------------------	CARD 5 ~ CARD 8	SEZIONE B
\coordinate (A) at ($(c5)-(0,\cardHeight*1.8)$);
    \gettikzxy{(A)}{\ax}{\ay}
    \gettikzxy{(in6bis)}{\inx}{\iny}
\coordinate (upR) at ($(b8)+(.4,\cardHeight-\offsetHeight)$);
\coordinate (upL) at ($(b5)+(-.4cm,\cardHeight-\offsetHeight)$);
    \gettikzxy{(upL)}{\upLx}{\upLy}
    \gettikzxy{(upR)}{\upRx}{\upRy}
\node[above right] at (upL) {\Large [ B ]};
\coordinate (newUpL) at (\upLx,\sezYup);
\coordinate (newUpR) at (\upRx,\sezYup);
\coordinate (downL) at ($(\upLx,\ay)-(0,1)$);
\coordinate (downR) at ($(\upRx,\ay)-(0,1)$);
\draw[dashed, line width = 1pt, rounded corners=10pt] (newUpL) -- (newUpR) -- (downR) -- (downL) -- (upL) -- cycle;
% ----------------------------	CARD 10 ~ CARD 11	SEZIONE C
\coordinate (upL) at ($(c8)+(-.4cm,\cardHeight-\offsetHeight)$);
\coordinate (upR) at ($(a11)+(.4,\cardHeight-\offsetHeight)$);
    \gettikzxy{(upL)}{\upLx}{\upLy}
    \gettikzxy{(upR)}{\upRx}{\upRy}
\node[above right] at (upL) {\Large [ C ]};
\coordinate (newUpL) at (\upLx,\sezYup);
\coordinate (newUpR) at (\upRx,\sezYup);
\coordinate (downL) at (\upLx,\sezYdown);
\coordinate (downR) at (\upRx,\sezYdown);
\draw[dashed, line width = 1pt, rounded corners=10pt] (newUpL) -- (newUpR) -- (downR) -- (downL) -- (upL) -- cycle;

%======================================================================














\end{tikzpicture}
\end{document}
